%% LyX 2.0.6 created this file.  For more info, see http://www.lyx.org/.
%% Do not edit unless you really know what you are doing.
%\documentclass[english]{article}
%\usepackage[LGR,T1]{fontenc}
%\usepackage[utf8]{inputenc}
%\usepackage{float}
%\usepackage{textcomp}
%\usepackage{subscript}
%
%\makeatletter
%
%%%%%%%%%%%%%%%%%%%%%%%%%%%%%%% LyX specific LaTeX commands.
%\DeclareRobustCommand{\greektext}{%
%  \fontencoding{LGR}\selectfont\def\encodingdefault{LGR}}
%\DeclareRobustCommand{\textgreek}[1]{\leavevmode{\greektext #1}}
%\DeclareFontEncoding{LGR}{}{}
%\DeclareTextSymbol{\~}{LGR}{126}
%%% Because html converters don't know tabularnewline
%\providecommand{\tabularnewline}{\\}
%
%\makeatother
%
%\usepackage{babel}
%\begin{document}
%
%\title{Review of Small Satellite Conference}
%
%
%\author{Javier Puig Navarro}
%
%
%\date{08/30/2013\pagebreak{}}
%
%\maketitle

\subsection{Constellations of Small Satellites}


\subsubsection{Planet Labs’ Remote Sensing Satellite System (2013)\cite{Marshall_PlanetLab}}

Planet Labs combines imaging capability with state-of-the-art big-data
and cloud-computing technologies to enable easy access to this data
by those who need it most. We describe the system under construction,
the technical specifications of the satellites, the ground station
network and the product. Satellite missions planned for launch in
2013.


\subsubsection{Can a Constellation of CubeSats Create a Capability? Satisfying Australia’s
Future Need for Multi-Spectral Imagery (2013)\cite{Stepan_ConstellationImagery}}

In 2011 Geosciences Australia released a document titled “Continuity
of Earth Observation Data for Australia - Operational Requirements
to 2015 for Lands, Coasts and Oceans.” This paper identifies future
satellite imagery requirements of the Australian economic region in
the 2015 timeframe. A need was identified for multi-spectral coverage
of the entire Australian landmass every day. This means medium spatial
resolution and high temporal resolution. This paper estimates the
gap in required imagery that will remain in 2015 when new public good
satellites are operational. It then describes a modification of the
6U cubesat system proposed that could supplement the public good systems.
Using commercial software Collection Planning \& Analysis Workstation
(CPAW), a model of the proposed space and ground segment is presented.
A key conclusion of this analysis is that a relatively small investment
in cubesats might meet a significant portion of Australia’s future
space imagery requirements.


\subsubsection{Simultaneous Multi-Point Space Weather Measurements using the Low
Cost EDSN CubeSat Constellation (2013)\cite{Gunderson_MultiptMeasureConstellation}}

The ability to simultaneously monitor spatial and temporal variations
in penetrating radiation above the atmosphere is important for understanding
both the near Earth radiation environment and as input for developing
more accurate space weather models. Due to the high variability of
the ionosphere and radiation belts, producing such a data product
must be done using high density multi-point measurements. The primary
scientific purpose of the Edison Demonstration of Smallsat Networks
(EDSN) mission is to demonstrate that capability by launching and
deploying a fleet of eight CubeSats into a loose formation approximately
500 km above Earth. The Energetic Particle Integrating Space Environment
Monitor (EPISEM) payload on EDSN will characterize the radiation environment
in low-earth orbit (LEO) by measuring the location and intensity of
energetic charged particles simultaneously over a geographically dispersed
area. The EDSN project is based at NASA’s Ames Research Center, Moffett
Field, California, and is funded by the Small Spacecraft Technology
Program (SSTP) in NASA’s Office of the Chief Technologist (OCT) at
NASA Headquarters, Washington. The EDSN satellites are planned to
fly late 2013 as secondaries on a DoD mission.


\subsubsection{Small Satellite Constellations for Earth Geodesy and Aeronomy (2013)}\cite{Conklin_GeoAero_Constellation}

Drag-free nano-satellite constellations can improve the sensitivity
and spatial and temporal resolution of Earth aeronomy and geodesy
measurements relative to a single satellite or a satellite pair. Multi-satellite
systems improve the frequency with which data can be collected for
a given location over the Earth. Drag-free satellites provide autonomous
precision orbit determination, accurately map the static and time
varying components of Earth's mass distribution, aid in our understanding
of the fundamental force of gravity, and will ultimately open up a
new window to our universe through the detection and observation of
gravitational waves. At the heart of this technology is a gravitational
reference sensor, which (a) contains and shields a free-floating test
mass from all non-gravitational forces, and (b) precisely measures
the position of the test mass inside the sensor. A feedback control
system commands thrusters to fly the “tender” spacecraft with respect
to the test mass. Thus, both test mass and spacecraft follow a pure
geodesic in spacetime. By tracking the position of a low Earth orbiting
drag-free satellite we can directly determine the detailed shape of
geodesics and, through analysis, the higher order harmonics of the
Earth’s geopotential. The commanded thrust, test mass position and
GPS tracking data can also be analyzed to produce the most precise
maps of upper atmospheric drag forces and, with additional information,
detailed models that describe the dynamics of the upper atmosphere
and its impact on all satellites that orbit the Earth.


\subsubsection{A 6U CubeSat Constellation for Atmospheric Temperature and Humidity
Sounding (2013)\cite{Sharmila_ContellationTempHumidity}}

We are currently developing a 118/183 GHz sensor that will enable
observations of temperature and precipitation profiles over land and
ocean. The 118/183 GHz system is well suited for a CubeSat deployment
as \textasciitilde{}10cm antenna aperture provides sufficiently small
footprint sizes (\textasciitilde{}25km). We will take advantage of
past and current technology developments at JPL viz. HAMSR (High Altitude
Microwave Scanning Radiometer), Advanced Component Technology (ACT’08)
to enable low-mass and low-power high frequency airborne radiometers.
In this paper, we will describe the design and implementation of the
118 GHz temperature sounder and 183 GHz humidity sounder instrument
on the 6U CubeSat. In addition, a summary of radiometer calibration
and retrieval techniques of the temperature and humidity will be discussed.


\subsection{Formation Flying Small Satellites}


\subsubsection{NASA's Four Spacecraft Magnetospheric Multiscale Mission (2013)\cite{Klumpar_4FF_spacecraft}}

The Magnetospheric Multiscale (MMS) mission is a Solar Terrestrial
Probes Program mission within NASA’s Heliophysics Division. The mission
consists of four identically-instrumented spin-stabilized satellites
that will be placed, with surgical accuracy, into the heart of Earth’s
magnetic reconnection regions. A mono-propellant propulsion system
with 12 thrusters on each 1250 kg spacecraft will execute both small
formation maintenance maneuvers and large apogee raise maneuvers –
achieving controlled interspacecraft separations down to 10 km at
mission-phased Earth-constellation distances eventually reaching 160,000
km. An attitude control system keeps the spacecraft to within \textpm{}0.5\textdegree{}
of the desired orientation using on-board closed loop maneuver control.

Scientifically, the MMS mission will utilize Earth’s natural plasma
laboratory to conduct the first sufficiently detailed measurements
to solve the microphysics of magnetic reconnection. Magnetic reconnection
is a fundamental plasma physical process that taps the energy stored
in a magnetic field and explosively converts it to particle kinetic
energy. This process occurs in star-accretion disk interactions, neutron
star magnetospheres, pulsar wind acceleration. It is implicated in
the acceleration of ultra-high-energy cosmic rays in active galactic
nuclei jets and occurs in solar flares and solar Coronal Mass Ejections
producing energetic charged particles that rain down on Earth’s environment.
Launch will be in October 2014 on an Atlas-V 421 Launch Vehicle.


\subsubsection{EDSN Update (2013)\cite{Yost_MultipointMeasurement}}

The Edison Demonstration of Smallsat Networks (EDSN) mission is one
of the first to be executed by the Small Spacecraft Technology Program
in NASA’s Space Technology Mission Directorate. EDSN’s mission objectives
are to demonstrate multi-point, repeatable science measurements, and
to create a basic satellite communications network using a swarm of
cubesats. Additionally, EDSN will build on the recently flown Phonesat
hardware as the EDSN spacecraft will use smartphone processors and
other sensors and devices that are have been demonstrated or are under
development within that project.


\subsubsection{The CanX-4\&5 Formation Flying Mission: A Technology Pathfinder for
Nanosatellite Constellations (2013)\cite{Bonin_FF_CanX-4&5}}

Future nano- and microsatellite constellations will require highly
precise absolute and relative position knowledge and control; intersatellite
communications; high-performance attitude determination and control
systems; and advanced, compact propulsion systems for orbit maintenance.
The dual spacecraft CanX-4\&5 mission - slated to launch in the late
2013 / early 2014 on India’s Polar Satellite Launch Vehicle (PSLV)
- will demonstrate all of these capabilities at the nanosatellite
scale: both as standalone subsystems, and in concert, to accomplish
autonomous formation flight with sub-meter relative position control
and centimeter-level relative position determination. Each spacecraft
is identical, and formation flight is enabled by each satellite having
a GPS receiver, on-board propulsion system, S-Band inter-satellite
link, and fine guidance and control (GNC) computer. The two spacecraft
will share on-board position, velocity, and attitude data wirelessly
over their intersatellite link, and one of the two spacecraft will
perform propulsive maneuvers to achieve and maintain a series of autonomous
formations. The technologies and algorithms used on CanX-4\&5 are
extensible to a broad range of missions and satellites at the nano-
and microsatellite scale, and this ambitious technology demonstration
will serve as a pathfinder for several formation flight and constellation
applications.


\subsubsection{Small Satellite Cluster Inter-Connectivity (2013)\cite{Radhika_FF_intercomunication}}

Small satellites can be launched in close formation flying patterns
to perform coordinated measurements of remote space missions. This
will allow a cluster of small satellites to be used to collect data
from multiple points and time, thereby providing spatial and temporal
resolutions that cannot be achieved with a single, conventional large
satellite. Our proposed system performance is evaluated using throughput,
access delay and end-to-end delay by running extensive simulations.
The throughput of the system is defi{}ned as the fraction of the total
simulation time used for a valid transmission. There are three different
formation flying patterns under study:

\begin{table}[H]


\caption{Small Satellite Inter-Connectivity Evaluation}


\begin{tabular}{|c|c|c|c|}
\hline 
Formation & Throughput (\%) & Access delay & End-to-end delay\tabularnewline
\hline 
\hline 
Leader-Follower & 23 & Less than Cluster & Less than Cluster\tabularnewline
\hline 
Cluster & 11 & More than Cluster & More than Cluster\tabularnewline
\hline 
Constellation & under study & under study & under study\tabularnewline
\hline 
\end{tabular}

\end{table}


The decision of which formation flying pattern has to be used depends
on the mission architecture, e.g. number of satellites, orbits, power,
etc. Inter-satellite communication eliminates the use of expensive
ground relay stations and ground tracking networks. It’s not necessary
to sink all the data from each of the small satellite to ground, thus
eliminating the need of intermediate ground stations for sending data.
The small satellite formation control problems, particularly, attitude
and relative position can be solved using inter-satellite communication
by exchanging the attitude and relative position information among
the small satellites. It can also provide timing synchronization.
This presentation aims to propose and validate inter-satellite communication
protocols for distributed small satellite networks.


\subsubsection{Enabling Radio Crosslink Technology for High Performance Coordinated
Constellations (2013)\cite{Voronka_FF_Crosslink}}

This paper describes the use of an advanced high-performance software
defined radio architecture to provide small satellites, including
CubeSats, with the ability to operate in coordinated constellations
and fractionated systems. While the advantages of small satellite
constellations are frequently discussed, the challenges of cooperative
operations in a constellation are often overlooked. We will discuss
the additional requirements that are often levied against a small
satellite constellation or fractionated system and how these requirements
can be efficiently addressed using software defined radio intersatellite
RF links. These capabilities will then be discussed relative to traditional,
uncoordinated solutions and how these capabilities enable classes
of missions that would otherwise be difficult to implement. In particular,
missions requiring cooperative, synchronized multi-point measurements
and real-time station keeping will be discussed. 


\subsubsection{From Single to Formation Flying CubeSats: An Update from the Delft
Programme (2013)\cite{Guo_FF_single2FF}}

Currently, most Cubesat missions are used for technology demonstrations
or education, which only explore the capability of an individual satellite.
However, the capability of Cubesats can be extremely enhanced by flying
a cluster of satellites. For example, several missions such as QB50
and OLFAR have been proposed for this purpose. This paper provides
an update of the Delfi programme of the Delft University of Technology
(TU Delft). Two CubeSats named Delta and Phi are going to be launched
to demonstrate autonomous formation flying. This paper consists of
three parts: 
\begin{itemize}
\item Overview of results and lessons learned from the development and the
mission implementations of the Delfi-C3 and Delfi-n3Xt satellites,
with emphasis on subsystem development and satellite design.
\item Differences and improvements from Delfi-C3 and Delfi-n3Xt towards
DelFFi. One of the important improvements is an advanced version of
the Attitude Determination and Control Subsystem (ADCS) with sensors
and actuators for 3-axis control.
\item Payloads of DelFFi that enable the autonomous formation flying. Here
the technology developments are threefold: 

\begin{itemize}
\item communicating, which concerns on inter-satellite communication and
ranging
\item processing, which utilizes multi-agent based artificial intelligence
technology for cooperative control
\item actuating, which performs formation control using a solid cool gas/micro-resistojet
combined propulsion system with high volume efficiency and a specific
impulse at 150s.
\end{itemize}
\end{itemize}

\subsubsection{Astronomical Antenna for a Space Based Low Frequency Radio Telescope
(2013)\cite{Quillen_FF_RadioTelescope}}

The Orbiting Low Frequency Antennas for Radio Astronomy (OLFAR) project is investigating an orbiting low frequency radio telescope. Due to strong ionospheric interference and Radio Frequency Interference (RFI) found at frequencies below 30 MHz, such an instrument is not feasible on Earth, hence the proposed solution of a swarm of autonomous nano-satellites sent to a remote location in space. On each satellite, the astronomical antenna consists of three orthogonal dipoles designed to work within the constraints of a nano-satellite. Due to mechanical constraints, the dipoles are not optimally integrated into the nano-satellites from an antenna point of view. Unfortunately, the operational band of 0:3 MHz to 30 MHz and the dimensions of the astronomical antenna of just under 5:0 m prohibit tests within the controlled environment of an anechoic chamber; ergo, a scale model is required. This work describes the design, simulation and measurement of such a scale model.

\subsubsection{Design of Nano-satellite Cluster Formations for Bi-Directional Reflectance
Distribution Function (BRDF) Estimations (2013)\cite{Nag_BRDF}}

The bidirectional reflectance distribution function (BRDF) of the
Earth’s surface describes the directional and spectral variation of
reflectance of a surface element. It is required for precise determination
of important geophysical parameters such as albedo. BRDF can be estimated
using reflectance data acquired at large 3D angular spread of solar
illumination and detector directions and visible/near infrared (VNIR)
spectral bands. This paper proposes and evaluates the use of nanosatellite
clusters in formation flight to achieve large angular spreads for
cheaper, faster and better estimations that will complement existing
BRDF data products. In this paper, the technical feasibility of this
concept is assessed in terms of various formation flight geometries
available to achieve BRDF requirements and multiple tradespaces of
solutions proposed at three levels of fidelity – Hill’s equations,
full sky spherical relative motion and global orbit propagation. Preliminary
attitude control requirements, as constrained by cluster geometry,
are shown to be achievable using CubeSat reaction wheels.


\subsubsection{Operations, Orbit Determination, and Formation Control of the AeroCube-4
CubeSats (2013)\cite{Gangestad_FF_AeroForces}}

Three satellites of the AeroCube-4 series built by The Aerospace Corporation
were launched in September 2012 from Vandenberg Air Force Base. These
satellites were each equipped with an on-board GPS receiver that provided
position measurements with a precision of 20 meters and enabled the
generation of ephemerides with meter-level accuracy. Each AeroCube
was also equipped with two extendable wings that altered the satellite’s
cross-sectional area by a factor of three. In conjunction with the
GPS measurements, high-precision orbit determination detected deliberate
changes in the AeroCube’s drag profile via wing manipulation. The
AeroCube operations team succeeded in using this variable drag to
re-order the satellites’ in-track configuration. A differential cross-section
was created by closing the wings of one satellite while the others’
remained open, and the relative in-track motion between two AeroCubes
was reversed. Over the course of several weeks, the satellites’ in-track
configuration was re-ordered, demonstrating the feasibility of CubeSat
formation flight via differential drag.


\subsection{Missions in which Formation Flying may be beneficial}


\subsubsection{Design and Analysis of a Nanosatellite Platform for Orbital Debris
Mitigation through Launch of Space Tether in Low Earth Orbits (2013)\cite{Asundi_SpaceDebris}}

Orbital debris disposal, particularly in low Earth orbits, has been
identified as a serious concern by NASA and other space agencies around
the world. To mitigate future overgrowth of the problem, guidelines
have been developed and proposed by these space agencies. However,
much of the research and development of established systems for de-orbiting
of satellites is focused on larger spacecraft, and efforts to reduce
the existing space debris in the low Earth orbit has been limited.
This paper will present a design and analysis of an approach to tether
existing space debris through a nanosatellite platform by solving
Lambert’s problem in real time. The platform will host a tether launching
system as its payload. The tether launching system is designed to
accommodate two Terminator Tape tethers from Tethers Unlimited. The
nanosatellite, when in orbit, will identify a nearby space debris
object with support from ground control. Through estimation of its
orbital position and prediction of orbital position of the space debris
object, the nanosatellite will perform a launch maneuver according
to Lambert’s problem to “sling” the onboard Terminator Tape onto the
debris object. The nanosatellite bus will largely be designed around
commercial off the shelf components.


\subsubsection{CubeSats – Have They Reached Their Explorer-1 Moment? (2013)\cite{Freeman_Exoplanet}}

The launch of Explorer-1 in February 1958 heralded the dawn of the
space age in the US. It was an appropriate response to the Soviet
Union’s Sputnik mission. This paper will put forward the argument
that cubesats are at or have passed their ‘Explorer-1 moment’. Missions
like the University of Michigan’s Radio Aurora Explorer, MIT/Draper
labs’ Exoplanetsat, JPL’s CHARM mission, are all recognizably science-driven
missions, designed to return valuable science data for heliophysics,
astrophysics and Earth Science. Rob Staehle at JPL has proposed interplanetary
cubesats, and others have suggested cubesats at Mars could yield unique
science data. It’s now possible to imagine a future – about 13 years
hence, in which constellations of cubesats are integral to observations
of the Earth system and climate change, dozens of cubesats are out
beyond Earth orbit, helping to access the hidden corners of our solar
system, monitor the Sun, and explore the Universe. This talk will
describe some of the efforts under way at the JPL to help enable this
future. 


\subsubsection{PanelSAR: A Smallsat Radar Instrument (2013)\cite{Dujin_Radar}}

A novel and low-cost solution for Synthetic Aperture Radar from space.
The paper focuses on a solution based on low power SAR principles
(FMCW/iFMCW), already proven in airborne SAR applications, that can
be embarked on a commercially affordable small satellite. The system
architecture is presented, showing how features such as modularity,
reliability, low power and low mass are achieved to provide a low-cost
end-to-end solution. Also the “new space” approach adopted in the
development of the smallsat miniSAR is addressed.


\subsubsection{Deployable Mirror for Enhanced Imagery Suitable for Small Satellite
Applications (2013)\cite{Champagne_deployablemirror}}

High spatial resolution imagery and large apertures go hand in hand
but small satellite volume constraints place a direct limit on monolithic
aperture mirror systems. Deployable optical systems hold promise of
overcoming aperture size constraints and greatly enhancing small satellite
imaging capabilities. The Space Dynamics Laboratory (SDL) is currently
researching deployable optics suitable for small spacecraft and has
developed a passively aligned deployable mirror. The team recently
built a proof-of-principle mirror and a single parabolic mirror segment
or “petal” measured for deployment repeatability. They measured elevation
(tilt) and azimuth (tip) angular alignment repeatability to be 0.6
arcseconds or 2.9 $\mu$rad (1 sigma) in each axis after a ten deployment
sequence. Excellent image quality is possible in the short wave infrared
(SWIR) to long wave infrared (LWIR) bands. 


\subsubsection{Versatile Structural Radiation Shielding and Thermal Insulation through
Additive Manufacturing (2013)\cite{Wrobel_Shielding}}

The use of Commercial Off-The-Shelf (COTS) components in SmallSat
platforms enable operational mission capabilities within affordable
program costs, although these components must either be isolated from
the space environment or be expected to have short lifetimes. In this
paper, we present results of our work adapting additive manufacturing
technologies to create multifunctional structures that provide tailored
isolation from the radiation and thermal environments of space. \textquotedbl{}Versatile
Structural Radiation Shielding\textquotedbl{} (VSRS™) are structures
that incorporate integral graded-Z radiation shielding, as well as
\textquotedbl{}Structural Multi-Layer-Shielding\textquotedbl{} (SMLI
™), components that incorporate integral thermal insulation. These
methods enable responsive fabrication of spacecraft structures with
complex geometries, such as avionics enclosures, conformal covers,
and even satellite 'exoskeleton' with tailored radiation shielding,
thermal isolation, heat dissipation, and EMI shielding.


\subsubsection{Collapsible Space Telescope (CST) for Nanosatellite Imaging and Observation
(2013)\cite{Elwood_CollapsibleTelescope}}

Nanosatellites have gained broad use within the university and scientific
communities for a variety of applications ranging from Space Weather,
Space Biology and Astrobiology. There is great interest to develop
high-quality nanosatellite imaging applications to support Earth Observations,
Astrophysics and Heliophysics Missions. NASA Ames Research Center
is developing a low cost, deployable telescope that, when coupled
with the appropriate imager, will provide high-resolution imaging
for Earth and Space Observations. The collapsible telescope design
is a Strain Deployable Ritchey Chrétien Cassegrain telescope that
can fit within the volume of 1U x 4U portion of a 6U nanosatellite
platform. Additional telescope optical prescriptions compatible with
the same deployable architecture are being explored. Our prototype
instrument backend is a remote sensing compact spectropolarimeter
with no moving parts, currently under development. The ability to
integrate a deployable Cassegrain telescope into a nanosatellite platform
matches desires outlined within the TA08 Remote Sensing Instruments/Sensors
Technical Area Roadmap and represents game changing technologies in
small satellite subsystems to include the potential for swarm missions
with distributed apertures.


\subsubsection{Asteroid Prospector (2013)\cite{Muelle_AsteroidProspector}}

This paper presents the overall design of a small reusable spacecraft
capable of ying to an asteroid from low earth orbit, operating near
the surface of the asteroid and returning samples to low earth orbit.
The spacecraft is in a 6U CubeSat form factor and designed to visit
near asteroids as far as 1.3 AU from the sun. Deep space missions
are traditionally large and expensive, requiring considerable manpower
for operations, use of the Deep Space network for navigation, and
costly but slow rad-hard electronics. Several new technologies make
this mission possible and affordable in such a small form factor:
a 3 cm ion engine from Busek for the low-thrust spirals, an autonomous
optical navigation system, precision miniature reaction wheels, high
performance and nontoxic green propellant (HGPG) thrusters, and Honeywell's
new Dependable Multiprocessor technology for radiation tolerance.
A complete spacecraft design is considered and the paper includes
details of the control and guidance algorithms. 


\subsubsection{FTS CubeSat Constellation Providing 3D Winds (2013) \cite{Wloszek_3Dwinds} }

A novel small satellite constellation utilizing a Fourier Transform
Spectrometer (FTS) instrument onboard 6U CubeSats would allow weather
forecasters to have unprecedented understanding of global tropospheric
wind observations from space; enabling more accurate, reliable, and
longer-term weather forecasts. Three FTS CubeSats flying in formation
and separated by a known time delay would provide cooperative measurements
and overlay scenes necessary to compile vertical profiles of the wind
field. A constellation of formation-flying FTS CubeSats would allow
measurement of the global wind field; providing unparalleled coverage
and allowing longer-term weather forecasts. This paper will describe
the recent advancements in CubeSat capabilities and future work required
to meet the objectives of the FTS mission. The innovative approach
the Exelis/University of Michigan team is taking to power, attitude
determination and control, communications, and constellation formations
will also be discussed.


\subsubsection{Real-Time Geolocation with a Satellite Formation (2013)\cite{Leiter_Geoloc}}

Space-borne geolocation with a small satellite formation could provide
accurate tracking of a Mars rover, a redundant navigation system in
a jammed GNNS environment, or a cost-effective system for autonomously
locating distress signals. In this study we demonstrate how a cluster
of two or three Low Earth Orbit (LEO) satellites performing sequential
time difference of arrival measurements could accurately determine
the position of a terrestrial source emitting electromagnetic pulses.
Whereas TDOA geolocation algorithms have been presented before, this
study provides a theoretical basis for achieving optimal positioning
performance, while solving for the initial position ambiguity through
recursive filtering techniques.


\subsection{CubeSats developed by students}


\subsubsection{The SwissCube: Results and Lessons Learned After 4 Years of Operations
in Space (2013)\cite{Rossi_SwissCube}}

SwissCube is the first Swiss built nanosatellite. Here we will present
results from more than 3 years of operations in space, including results
from the payload. In this paper we present a general description of
the hardware used for SwissCube, giving a particular attention to
the tests done before and after flight, describing the payload mounted
and the results achieved. We will detail our analysis of the Attitude
Control and Determination subsystem, which is based on a B-dot controller
using magento-torquers, gyros, magneto-meters, sun-sensors and thermometers.



\subsection{Interesting papers}


\subsubsection{Findings of the KECK Institute for Space Studies Program on Small
Satellites: A Revolution in Space Science (2013)\cite{Norton_KISS}}

The Keck Institute for Space Studies (KISS) is a \textquotedbl{}think
and do tank\textquotedbl{} established at Caltech where a small group
of not more than 30 persons interact for a few days to explore various
frontier topics in space studies. The primary purpose of KISS is to
develop new planetary, Earth, and astrophysics space mission concepts
and technology. The first workshop identified novel mission concepts
where stand-alone, constellation, and fractionated small satellite
systems can enable new targeted space science discoveries in heliophysics,
astrophysics, and planetary science including NEOs and small bodies.
The second workshop then identified the technology advances necessary
to enable these missions in the future. In the following we review
the outcome of this study program as well as the set of recommendations
identified to enable these new classes of missions. 


\subsubsection{The Next Little Thing: Femtosatellites (2013)\cite{Janson_Femtosats}}

The original CubeSat vision was to enable simple, meaningful missions
that universities could undertake. CubeSats were later adopted by
industry and government agencies. The community focus on miniaturization
has been costly in our endeavor to do more with less. Femtosatellites,
defined as having a mass less than 100 grams, turn this scenario on
its head by forcing a do less with more mentality; individual spacecraft
will be less capable, but coordinated operation of massively distributed
femtosatellites can achieve the required overall mission capability.



\subsubsection{Autonomous Assembly of a Reconfiguarble Space Telescope (AAReST)
– A CubeSat/Microsatellite Based Technology Demonstrator (2013)\cite{Underwood_ReconfigTelescope}}

Future space telescopes with diameter over 20 m will require in-space
assembly. High-precision formation flying has very high cost and may
not be able to maintain stable alignment over long periods of time.
We believe autonomous assembly is a key enabler for a lower cost approach
to large space telescopes. The mission will involve two 3U CubeSat-like
nanosatellites (“MirrorSats”) each carrying an electrically actuated
adaptive mirror, and each capable of autonomous un-docking and re-docking
with a small central “9U” class nanosatellite core, which houses two
fixed mirrors and a boom-deployed focal plane assembly. 


\subsubsection{NASA’s GRAIL Spacecraft Formation Flight, End of Mission Results,
and Small-Satellite Applications (2013) \cite{Edward_GRAIL}}

The Gravity Recovery and Interior Laboratory (GRAIL) mission was composed
of twin spacecraft tasked with precisely mapping the gravitational
field of Earth’s Moon. GRAIL science collection required that the
two spacecraft operate in the same orbit plane and with precise relative
separation and pointing, which evolved through the primary and extended
mission Science phases.


%\subsection{Conclusions}

%Formation Flight CubeSats are capable of great contributions to science
%and technology as:
%\begin{enumerate}
%\item To avoid spacial and temporal ambiguities
%
%\begin{itemize}
%\item Multipoint measurement of plasma density
%\item Multipoint measurement of electric y magnetic fields (geomagnetic
%storms)
%\item Multipoint measurement of proton and electron fluxes
%\item Planetary Atmospheric espectroscopy
%\end{itemize}
%\item Earth monitoring (a constellation should be enough)
%
%\begin{itemize}
%\item Agriculture and vegetation management
%\item Weather prediction models
%\item Oceanographye
%\item Disasters assesment
%\item Measurements of compounds on the atmosphere
%\item Pollution
%\item CO\textsubscript{2} or H\textsubscript{2}O cycle (or any other compound)
%\item Doppler sensing of winds
%\item Temperature sensing
%\end{itemize}
%\item Testbed for mobile phone electronics for satellite intercommunications
%or satellite-ground communications
%\item To carry a web and remove space junk
%\item To maneuver with solar sails
%\item To explore asteroids (spider cubesat swarm, could be the basis to
%industrialize asteroids)
%\item Automated 3D mapping (google maps or any other applications)
%\item Observation of gravitational waves
%\item Interferometry
%
%\begin{itemize}
%\item Exoplanet detection
%\item Infrared telescopes
%\item Gamma-ray telescopes
%\item X-ray telescopes
%\item Visual spectrum telescopes
%\item Radio Telescopes
%\end{itemize}
%\end{enumerate}
%\bibliographystyle{IEEEtran}
%\bibliography{Formation_Flying_Missions_Javier}

%\end{document}
