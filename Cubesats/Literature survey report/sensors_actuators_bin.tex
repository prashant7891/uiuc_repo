<<<<<<< HEAD
%%% Comment:
%\documentclass[english]{article}
%\usepackage[T1]{fontenc}
%\usepackage[latin9]{inputenc}
%\usepackage{babel}
%\begin{document}
%
%\title{Sensors and Actuators}
%
%
%\author{Hong-Bin Yoon}
%
%\maketitle

\begin{itemize}
\item 3U Atmospheric Satellite. Uses passive microwave spectrometer.  reaction wheels, magnetorquers, Earth horizon sensors\cite{Ref:Blackwell12}. 

\item 3U satellite. Demonstrate de-orbiting technology for LEO(drag sail). \cite{Ref:Shmuel12}. 

\item 3.7-kg mass nanosatellite ejected from the Space Shuttle Atlantis during the final STS-135 mission on July 20, 2011. PSSCT-2 had a three-axis attitude control system to enable firing of solid rockets for orbit raising, pointing of solar cells normal to the sun for on-orbit performance monitoring, and pointing of a GPS antenna in the anti-flight direction for radio-occultation measurements.Hybrid Propulsion system. Pressure vessels are made using new high performance carbon-fiber reinforced polymer material. Hybrid propulsion is safer than solid, simpler than liquid.\cite{Ref:Janson12}.

\item Miniature ion electrospray thrusters. effort fits within 1/3 of one 1U cubesat and is designed to provide fine three-axis attitude control and precision thrusting, to deliver a total Delta-V in excess of 200 m/sec to 3U cubesats\cite{Ref:Dushku12}. 

\item Mission proposal to test Differential Optical Shadow Sensor (DOSS) on 2U Sat   -position sensor\cite{Ref:Martel12}. 

\item A stellar gyroscope for ADCS (higher update rate star tracker)\cite{Ref:Zoellner12}. 

\item Two MEMS (Micro Electro Mechanical Systems) components suitable for small satellite propulsion applications. \cite{Ref:Rawashdeh12}. 
\item Two cubesats with miniaturized propulsion system are used for the mission\cite{Ref:Sundaramoorthy10}. 

\item Prototype SPT consists of 36 individual thrusters with chambers 1.5 mm in diameter and approximately 3 mm in length. 0.15 - 0.28 mN thrust range.\cite{Ref:Sathiyanathan10}. 
\item A new method for measuring performance of a micro-thruster with minimal calibration measurements.\cite{Ref:Chang08}. \\\\
overview of ultra-short baseline GPS attitude determination experiments\cite{Ref:Bageshwar06}. 
\item Presents GPS based ADS. uses three GPS antennas to achieve subcentimeter accuracy.\cite{Ref:Gershman06}. 

\item A survey is presented of existing technology driven by the following primary requirementsIbit less than 1mNs, mass less than 1kg, and size less than one 10cm cube unit.\cite{Ref:Storck06}. 

\item ATSBTM CMOS horizon sensor has an RMS accuracy of better than 0.1$^o$, a mass of 560g and consumes only 550mW when imaging.\cite{Ref:Bahar06}. 

\item Discuss Vision-Based Attitude and Formation Determination System (VBAFDS), including hardware requirements, algorithm development, and simulation results\cite{Ref:Rogers04}. 

\item University of Illinois 2-cube CubeSat used vacuum arc thruster (VAT) propulsion system.  4 x 4 x 4 cm and 150 g.  1$\mu$N-s/pulse\cite{Ref:Rysanek02}. 
\end{itemize}




%\bibliographystyle{IEEEtran}
%\bibliography{sensors_actuators_bib_bin}

%\end{document}
=======
%%% Comment:
%\documentclass[english]{article}
%\usepackage[T1]{fontenc}
%\usepackage[latin9]{inputenc}
%\usepackage{babel}
%\begin{document}
%
%\title{Sensors and Actuators}
%
%
%\author{Hong-Bin Yoon}
%
%\maketitle

\begin{itemize}
\item 3U Atmospheric Satellite. Uses passive microwave spectrometer.  reaction wheels, magnetorquers, Earth horizon sensors\cite{Ref:Blackwell12}. 

\item 3U satellite. Demonstrate de-orbiting technology for LEO(drag sail). \cite{Ref:Shmuel12}. 

\item 3.7-kg mass nanosatellite ejected from the Space Shuttle Atlantis during the final STS-135 mission on July 20, 2011. PSSCT-2 had a three-axis attitude control system to enable firing of solid rockets for orbit raising, pointing of solar cells normal to the sun for on-orbit performance monitoring, and pointing of a GPS antenna in the anti-flight direction for radio-occultation measurements.Hybrid Propulsion system. Pressure vessels are made using new high performance carbon-fiber reinforced polymer material. Hybrid propulsion is safer than solid, simpler than liquid.\cite{Ref:Janson12}.

\item Miniature ion electrospray thrusters. effort fits within 1/3 of one 1U cubesat and is designed to provide fine three-axis attitude control and precision thrusting, to deliver a total Delta-V in excess of 200 m/sec to 3U cubesats\cite{Ref:Dushku12}. 

\item Mission proposal to test Differential Optical Shadow Sensor (DOSS) on 2U Sat   -position sensor\cite{Ref:Martel12}. 

\item A stellar gyroscope for ADCS (higher update rate star tracker)\cite{Ref:Zoellner12}. 

\item Two MEMS (Micro Electro Mechanical Systems) components suitable for small satellite propulsion applications. \cite{Ref:Rawashdeh12}. 
\item Two cubesats with miniaturized propulsion system are used for the mission\cite{Ref:Sundaramoorthy10}. 

\item Prototype SPT consists of 36 individual thrusters with chambers 1.5 mm in diameter and approximately 3 mm in length. 0.15 - 0.28 mN thrust range.\cite{Ref:Sathiyanathan10}. 
\item A new method for measuring performance of a micro-thruster with minimal calibration measurements.\cite{Ref:Chang08}. \\\\
overview of ultra-short baseline GPS attitude determination experiments\cite{Ref:Bageshwar06}. 
\item Presents GPS based ADS. uses three GPS antennas to achieve subcentimeter accuracy.\cite{Ref:Gershman06}. 

\item A survey is presented of existing technology driven by the following primary requirementsIbit less than 1mNs, mass less than 1kg, and size less than one 10cm cube unit.\cite{Ref:Storck06}. 

\item ATSBTM CMOS horizon sensor has an RMS accuracy of better than 0.1$^o$, a mass of 560g and consumes only 550mW when imaging.\cite{Ref:Bahar06}. 

\item Discuss Vision-Based Attitude and Formation Determination System (VBAFDS), including hardware requirements, algorithm development, and simulation results\cite{Ref:Rogers04}. 

\item University of Illinois 2-cube CubeSat used vacuum arc thruster (VAT) propulsion system.  4 x 4 x 4 cm and 150 g.  1$\mu$N-s/pulse\cite{Ref:Rysanek02}. 
\end{itemize}




%\bibliographystyle{IEEEtran}
%\bibliography{sensors_actuators_bib_bin}

%\end{document}
>>>>>>> f6d6d10ca248bedea43366ea776bdd73a3528419
