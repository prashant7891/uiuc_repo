\documentclass[11pt,a4paper]{article}
\usepackage[utf8]{inputenc}
\usepackage{amsmath}
\usepackage{amsfonts}
\usepackage{amssymb}
\usepackage{fullpage}
\begin{document}
\begin{center}
{\Huge New Ideas: Giri}
\end{center}

\section{Space Wake studies}
The vacuum that has been generated by the best vacuum generators on ground have gone only go upto a particular level of purity. However, we can use the naturally available vacuum in space. The problem is that the vacuum available in LEO is not very pure. High purity vacuum can be obtained in higher orbits, but it is not feasible to reach those with CubeSats. This idea is that we can have a CubeSat with an inflatable disk-like structure orbit around the earth in LEO. Studies show that this produces a very highly pure vacuum. We can have the other CubeSats follow in this wake to conduct any required experiments. Formation flying is useful for this mission, it can be performed in LEO and 4-5 CubeSats might be ideal for it (depending on the application).

\section{Open source CubeSat Testbed}

We can have a mission in which we launch a set of 4-5 CubeSats in LEO flying in a basic formation. The architecture of these CubeSats will be open to the public so that anyone can write code for it. The limitations and the capabilities of the CubeSats will be clearly listed and everyone would have to follow it. They can send us the code and we can then upload it to the CubeSats which will perform them in real time. This is like a tested for CubeSat algorithms. A lot of people now would have algorithms that they would like to test, but might not have access to satellites. This is again ideal for us as it's in LEO, 4-5 CubeSats will be sufficient and formation flying is essential. The second advantage of this is the additional source of funding which can be charged for people to test their code on our testbed. The third advantage is that it will just be a technology demonstration from our end without haveing to worry much about the physcics or science behind the missions, thus allowing us to define the requirements. The problem is that, the lifetime of the testbed would be as of now (~1-2 months). However, it is a reasonable assumption that this would increase as new technologies are developed. 

\section{Using Space Tethers}




\end{document}