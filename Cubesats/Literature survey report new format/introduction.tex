\section{Introduction}
In a world where lack of space is becoming an increasingly problematic issue to deal with, miniaturization is becoming more of  a necessity rather than an option. With sensors and actuators reaching small scales, it is now possible to actually design pico and nano satellites to meet certain required scientific goals. In this report, we are primarily concerned with nano satellites or \href{http://www.cubesat.org/images/developers/cds_rev12.pdf}{CubeSats} with sizes ranging from 1U(10cm x 10cm x 10cm) to 6U(60cm x 10cm x 10cm). Since CubeSats are small, it is possible to launch multiple CubeSats with a single scientific mission. Thus, using formation flying CubeSats, it is possible to perform missions which, earlier, were either too difficult to do or required much bigger satellites. \\

The objective of this literature survey is three-fold: 
\begin{enumerate}
\item To uncover and understand the existing single, as well as multiple, satellite missions and some proposed concepts.
\item To enumerate the various actuators and sensors that are available today or will be available in the near future (with an acceptable Technology Readiness Level) and to analyse if they would satisfy our mission requirements. 
\item To propose new missions based on formation flying of CubeSats.
\end{enumerate}


These missions have been categorized into five different classes as \href{http://solarsystem.nasa.gov/multimedia/download-detail.cfm?DL_ID=742}{Planetary Science}, \href{http://www.nap.edu/catalog.php?record_id=11820}{Earth Science}, \href{http://www.nap.edu/catalog.php?record_id=12951}{Astrophysics}, \href{http://www.nap.edu/catalog.php?record_id=13060}{Heliophysics} and Technology Demonstrations. There is also two charts presented at the end of these five sections, which give an idea of the various concepts that can be realized along with their scientific requirements. 
