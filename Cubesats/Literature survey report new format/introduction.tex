\section{Introduction}
In a world where lack of space is becoming an increasingly problematic issue to deal with, miniaturization is becoming more of  a necessity rather than an option. With sensors and actuators reaching small scales, it is now possible to actually design pico and nano satellites to meet certain required scientific goals. In this report, we are primarily concerned with nano satellites or \href{http://www.cubesat.org/images/developers/cds_rev12.pdf}{CubeSats} with sizes ranging from 1U(10cm x 10cm x 10cm) to 6U(60cm x 10cm x 10cm). Since CubeSats are small, it is possible to launch multiple CubeSats with a single scientific mission. Thus, using formation flying CubeSats, it is possible to perform missions which, earlier, were either too difficult to do or required much bigger satellites. \\

The objective of this literature survey is three-fold: 
\begin{enumerate}
\item To uncover and understand the existing single, as well as multiple, satellite missions and some proposed concepts.
\item To enumerate the various actuators and sensors that are available today or will be available in the near future (with an acceptable Technology Readiness Level) and to analyse if they would satisfy our mission requirements. 
\item To propose new missions based on formation flying of CubeSats.
\end{enumerate}


These missions have been categorized into five different classes as \href{http://solarsystem.nasa.gov/multimedia/download-detail.cfm?DL_ID=742}{Planetary Science}, \href{http://www.nap.edu/catalog.php?record_id=11820}{Earth Science}, \href{http://www.nap.edu/catalog.php?record_id=12951}{Astrophysics}, \href{http://www.nap.edu/catalog.php?record_id=13060}{Heliophysics} and Technology demonstrations. There is also two charts presented at the end of these five sections, which give an idea of the various concepts, distributed across all these categories, that can be realized along with their scientific requirements. 

Before explaining each of the missions individually, a graph, showing all the missions that have been mentioned in this report, has been plotted in Figure 1. The vertical axis has the number of satellites that are required to perform that particular mission. The horizontal axis has been split according to whether it can be performed using a single spacecraft or whether formation flying or constellations are required to perform the same. They have also been colour coded according to the mission class, i.e., into the five categories that were specified above. 

Even though there are exceptions, it can be seen from Figure 1 that most of the existing missions are clustered towards the origin, i.e., they are missions which can mostly be done by single satellites. Thus there is a need for multiple satellite missions which can perform different missions or an existing mission more efficiently. Also, formation flying satellites have an advantage over constellations in many missions. Hence, most of our mission proposals have been aimed towards the formation flying region of the plot with 4-10 satellites.  


\scalefig{cost_function.png}{1}{Mission Categorization}
\newpage
{\bf \underline {Key}}: \\

\begin{tabular}{p{20cm}}
 1 - CADRE (Section~\ref{CADRE}) \\
 2 - EXOCUBE (Section~\ref{exocube}) \\
 3 - Earthquake prediction using Ionospheric monitoring (Section~\ref{predicting_earthquakes}) \\
 4 - Ionosphere reaction to storms (Section~\ref{iono_storm_reaction}) \\
 5 - Atmospheric Plasma Depletion to predict GPS outages (Section~\ref{atm_plasma}) \\
 6 - Earth imaging (Section~\ref{imaging_science}) \\
 7 - Gamma rays of thunderstorms (Section~\ref{gamma_storm}) \\
 8 - Space-based ocean monitoring (Section~\ref{ocean_monitor}) \\
 9 - Map of Earth's electric field (Section~\ref{elec_field}) \\
10 - Raman spectroscopy (Section~\ref{raman}) \\
11 - Magnetosphere sampling (Section~\ref{vol_magnetosphere}) \\
12 - Australian multi-spectral imagery (Section~\ref{australia_imagery}) \\
13 - Earth geodesy and astronomy (Section~\ref{geodesy}) \\
14 - Atmospheric temperature and humidity sounding (Section~\ref{temp_sounding}) \\
15 - EDSN (Section~\ref{edsn}) \\
16 - Bi-directional reflectance distribution function (Section~\ref{brdf}) \\
17 - Orbital debris mitigation (Section~\ref{debris_mitigation}) \\
18 - Collapsible space telescope (Section~\ref{cst}) \\
19 - Fourier transform spectrometer (Section~\ref{fts}) \\
20 - Mineral mapping of asteroids (Section~\ref{min_map}) \\
21 - Solar system escape (Section~\ref{ss_escape}) \\
\end{tabular}
\begin{tabular}{p{20cm}}
22 - Radio quite lunar CubeSat (Section~\ref{rqlc}) \\
23 - Tracking asteroids (Section~\ref{track_ast}) \\
24 - AeroCube - 4 (Section~\ref{aerocube}) \\
25 - Asteroid Prospector (Section~\ref{ast_prospect}) \\
26 - Real time Geolocation (Section~\ref{geolocation}) \\
27 - NASA's GRAIL (Section~\ref{grail}) \\
28 - Pinpointing gamma ray bursts (Section~\ref{gamma_burst}) \\
29 - Interferometry and SAR (Section~\ref{sar}) \\
30 - Studying sub-dwarf stars (Section~\ref{dwarf}) \\
31 - Space based low frequency radio telescope (Section~\ref{astro_ant}) \\
32 - PanelSAR (Section~\ref{panelsar}) \\
33 - Deployable mirror for enhanced imagery (Section~\ref{deploy_mirror}) \\
34 - AAReST (Section~\ref{aarest}) \\
35 - Colorado student space weather experiment (Section~\ref{colorado}) \\
36 - Solar polar imager (Section~\ref{solar_polar}) \\
37 - Earth-Sun Sunward-of-L1 solar monitor (Section~\ref{l1}) \\
38 - Testing tether deployment (Section~\ref{tether}) \\
39 - CanX-4\&5 (Section~\ref{canx4}) \\
40 - Division of labor in satellite (Section~\ref{labour})  \\
41 - Space Advertising (Section~\ref{ad}) \\
42 - Orbiting marker in space (Section~\ref{marker}) \\
43 - Graveyard orbit transfer service (Section~\ref{grave}) \\
44 - Optical communication mission (Section~\ref{op_comm}) \\
45 - Satellite service in maintenance and repair (Section~\ref{mro}) \\
46 - 3D printing in space (Section~\ref{3d}) \\
47 - Tracking gamma ray bursts (Section~\ref{grb}) \\
48 - Ionospheric tomography (Section~\ref{tomography}) \\
49 - Inflatable De-orbit device (Section~\ref{inflatable}) \\
50 - Variable range coronograph (Section~\ref{coronograph}) \\
51 - Tracking asteroid belts (Section~\ref{ast_belt}) \\
52 - Sun Energy collector (Section~\ref{energy_collector}) \\
53 - Space wake studies Section~\ref{space_wake}) \\
54 - Open source CubeSat testbed Section~\ref{testbed}) \\
\end{tabular}

\vspace{1cm}
