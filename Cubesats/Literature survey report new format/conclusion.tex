\section{Conclusions}
Formation flight CubeSats are capable of great contributions to science
and technology such as:
\begin{enumerate}
\item To avoid spacial and temporal ambiguities

\begin{itemize}
\item Multipoint measurement of plasma density
\item Multipoint measurement of electric and magnetic fields (geomagnetic storms)
\item Multipoint measurement of proton and electron fluxes
\item Planetary Atmospheric spectroscopy
\end{itemize}
\item Earth monitoring (a constellation should be enough)

\begin{itemize}
\item Agriculture and vegetation management
\item Weather prediction models
\item Oceanography
\item Disasters assessment
\item Measurements of compounds on the atmosphere
\item Pollution
\item CO\textsubscript{2} or H\textsubscript{2}O cycle (or any other compound)
\item Doppler sensing of winds
\item Temperature sensing
\end{itemize}
\item Testbed for mobile phone electronics for satellite intercommunications
or satellite-ground communications
\item To carry a web and remove space junk
\item To maneuver with solar sails
\item To explore asteroids (spider cubesat swarm, could be the basis to
industrialize asteroids)
\item Automated 3D mapping (google maps or any other applications)
\item Observation of gravitational waves
\item Interferometry

\begin{itemize}
\item Exoplanet detection
\item Infrared telescopes
\item Gamma-ray telescopes
\item X-ray telescopes
\item Visual spectrum telescopes
\item Radio Telescopes
\end{itemize}
\end{enumerate}