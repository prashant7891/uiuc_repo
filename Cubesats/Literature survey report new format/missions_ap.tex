\section{Astrophysical Missions}

\subsection{Pinpointing the Source of Gamma Ray Bursts (Concept)}
\label{gamma_burst}

The formation could be used to source GRBs through precise triangulation. If researchers could receive the measurements of inter-satellite distance and GRB incident time accurately enough, they could potentially increase the accuracy of GRB detection and positioning. \cite{Ref:Dill}. This has since been proven unnecessary, there are sufficient satellites in orbit to do this, through the use of gamma ray detectors and UV detectors. The UV detectors can do the positioning that would have required a formation with only gamma ray detectors. 

\subsection{Interferometry and Synthetic Aperture Radar Formation Flying}
\label{sar}

A formation of >2 CubeSats will work together to create a digital
terrain model or study surface deformation. A cross-track pendulum
formation is easier to isolate the crosstrack and along-track components. A cartwheel formation, however, reduces the height errors \cite{Ref:Peterson}. The application is widely studied for formations. 

\subsection{Studying Sub-dwarf Stars Using a Small Telescope}
\label{dwarf}
A satellite-bourne telescope would be used to study distant sub-dwarf stars, to set the lower limit on the age of ``metal-poor sub-dwarf'' stars to help establish an age for the universe. The attitude control system would need to be accurate to within 30 arc-seconds \cite{Ref:Carroll2}. Another possible mission would be imaging star fields \cite{Ref:Nicholas}. To make the formation relevant, each satellite would need a different type of instrument to further the science. 

\subsection{Astronomical Antenna for a Space Based Low Frequency Radio Telescope}
\label{astro_ant}
A great atmospheric interference occurs for low radio frequencies,
specially bellow 30 MHz. For that reason low frequency radio telescopes
are not feasible on the surface of the Earth. Therefore the only way
to collect reliable data in that frequency is by placing the instruments
in space. The Orbiting Low Frequency Antennas for Radio Astronomy
(OLFAR) consists of a swarm of nano-satellites, each of them with
three dipoles that satisfy the space restrictions in cubesats, intended
to build a low-frequency distributed radio telescope in space. This
paper describes the design, simulations, testing and measurements
of a scale model of the system. \cite{Quillen_FF_RadioTelescope}

\subsection{PanelSAR: A Smallsat Radar Instrument}
\label{panelsar}
PanelSAR describes a solution to implement a low-cost and low power consuming synthetic aperture radar in a small satellite. The study states that a low power SAR, which has already been proved in aircraft, and several other features of the system, such as modularity, reliability and low mass are key to reach a low cost solution.\cite{Dujin_Radar}

\subsection{Deployable Mirror for Enhanced Imagery Suitable for Small Satellite Applications}
\label{deploy_mirror}

Small satellite volume restrictions are a clear drawback for large
aperture monolithic mirrors for telescopes. Consequently, high spatial
resolution can not be achieved with a single mirror. One solution
to this problem is the development of deployable optical systems that
could surpass the performance of monolithic mirrors and greatly improve
the capabilities of small satellites. A passively aligned deployable
mirror is under development by the Space Dynamics Laboratory (SDL) and
they have built one of the ``petals'' that conform the deployable
mirror. The spectrum they are interested in varies from short wave
infrared to long wave infrared due to the image quality obtained.\cite{Champagne_deployablemirror}

\subsection{Autonomous Assembly of a Reconfigurable Space Telescope (AAReST) \textendash{} A CubeSat/Microsatellite Based Technology Demonstrator}
\label{aarest}
This study has been reported here because it remarks the disadvantages
of formation flying satellites and proposes an alternative to formation
flight, namely autonomous assembly. The paper states that autonomous assembly
is cheaper and more efficient when building space telescopes with
large apertures. In addition, the study claims that it is difficult
to maintain a stable alignment among spacecrafts in flight formation.
The mission developed in the study contains two 3U cubesats, each
of them would operate an electrically actuated adaptive mirror. In
addition, there would be a central 9U cubesat core, housing two fixed
mirror, where the 3U cubesats could dock and un-dock.\cite{Underwood_ReconfigTelescope}