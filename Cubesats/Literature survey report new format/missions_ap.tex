\section{Astrophysical Missions}

\subsection{Pinpointing the Source of Gamma Ray Bursts}

The formation could be used to source GRBs through precise triangulation. If we could get the measurements of inter-satellite distance and GRB incident time accurately enough, we could potentially increase the accuracy of GRB detection and positioning. \cite{Ref:Dill}. This has since been proven unnecessary, there are sufficient satellites in orbit to do this, through the use of gamma ray detectors and UV detectors. The UV detectors can do the positioning that would have required a formation with only gamma ray detectors. 

\subsection{Interferometry and Synthetic Aperture Radar Formation Flying}

A formation of >2 CubeSats will work together to create a digital
terrain model or study surface deformation. A cross-track pendulum
formation is easier to isolate the crosstrack and along-track components. A cartwheel formation, however, reduces the height errors \cite{Ref:Peterson}. The application is widely studied for formations. 

\subsection{Studying Sub-dwarf Stars Using a Small Telescope}

A satellite-bourne telescope would be used to study distant sub-dwarf stars, to set the lower limit on the age of ``metal-poor sub-dwarf'' stars to help establish an age for the universe. The attitude control system would need to be accurate to within 30 arc-seconds cite{Ref:Carroll2}. Another possible mission would be imaging star fields \cite{Ref:Nicholas}. To make the formation relevant, each satellite would need different a different type of instrument to further the science. 

