\section{Planetary Science Missions}

\subsection{Mineral Mapping of Asteroids (Concept)}
The proposed mission overview is a single 6U CubeSat launched on a GEO satellite 
or Mars-bound mission as a secondary payload. There is a solar sail to reach near Earth asteroids. The proposed science objectives is to map surface composition of ~3 asteroids at 1-20 m spatial resolution.

Required instrumentation: ~spatial IFOV of 0.5 mrad, spatial sampling 0.5 m -10 m depending on the encounter range, Spectral sampling 10 nm, Imaging Spectrometer, 0.4 – 1.7 $\mu$m. Perhaps extend to 2.5 $\mu$m w/ HOT-BIRD or other advanced detector and achievable cooling. \href{http://kiss.caltech.edu/cosponsored/cubesat2012/presentations/staehle-interplanetary-cubesat-missions.pdf}{Link to presentation given by Robert Staehle}

\subsection{Solar system escape (Concept)}
The plan is to use a large area/ low mass spacecraft for high speed trajectory with low perihelion which would explore interplanetary environment, heliosheath and perhaps heliopause. It is also aimed to test communications, power, pointing and miniaturized instrument technologies.  \href{http://kiss.caltech.edu/cosponsored/cubesat2012/presentations/staehle-interplanetary-cubesat-missions.pdf}{Link to presentation given by Robert Staehle}

Required Instrumentation: Plasma, solar wind, Energetic particles \& cosmic rays, Magnetometer, Cameras to observe sail interaction with environment.


\subsection{Radio Quiet Lunar CubeSat (Concept)}
The aim is to assess radio quiet volume in shielded zone behind the Moon for future 21 cm cosmology missions. The proposed science mission is to find the usable volume behind the Moon for high sensitivity 21 cm cosmology observations which determines utility of lunar surface vs. orbiting missions.  \href{http://kiss.caltech.edu/cosponsored/cubesat2012/presentations/staehle-interplanetary-cubesat-missions.pdf}{Link to presentation given by Robert Staehle}

\subsection{Tracking Asteroids and Satellite Debris }

A formation of satellites would search for Earth-Approaching Asteroids
and potentially hazardous debris satellites using a small imaging
telescope \cite{Ref:Carroll}. One study suggests an orbit of 6000-40000
km \cite{Ref:Leitch}. 

Similarly, space debris of 1-10 cm is difficult to track using current
methods and can be very dangerous to things like solar arrays and
radiators \cite{Ref:Guerrero}. One study suggests the use of optical
and radar telescopes to track the small debris \cite{Ref:Tolkachev}.

\subsection{The CanX-4\&5 Formation Flying Mission: A Technology Pathfinder for Nanosatellite Constellations}

Accuracy and miniaturization of propulsion systems, attitude determination
sensors, control systems, inter-satellite communication systems and
relative position sensors is one of the key aspects for future small
satellite missions. The two spacecrafts Can X-4\&5 have been designed
as a technology demonstrator of these capabilities as independent
spacecraft but also to cooperate in flight formation. Both spacecrafts
will have access to the state vector of the other spacecraft wirelessly.
In addition, the pair of small satellite has been equipped with a
sub-meter relative position control, a centimeter-accuracy relative
position determination, a GPS receiver, an on board computer and an
inter-satellite communication system.\cite{Bonin_FF_CanX-4&5}

\subsection{Operations, Orbit Determination, and Formation Control of the AeroCube-4 CubeSats}

The company Aerospace Corporation built three satellites of the AeroCube-4
series that were launched in 2012. Each of these satellites was able
to estimate its position with 20m of accuracy by means of a GPS receiver
installed in each spacecraft. In addition, each satellite was equipped
with extendable wings that allowed variations in the cross-sectional
area of the spacecraft. This two features, the high precision orbital
positioning and the variable wings, allowed to measure the deliberate
changes in the drag profile caused by the different configurations
of the wing. The purpose of the project was to achieved formation
flight via wing manipulation and indeed the team succeeded reordering
the satellites over the course of several weeks.\cite{Gangestad_FF_AeroForces}

\subsection{Asteroid Prospector}

This paper contains the design of a small reusable spacecraft that
is able to go to asteroids from LEO as long as they are closer than
1.3 AU from the sun. The paper includes details on the several new
technologies that are needed to make this mission possible such as:
a 3 cm ion engine from Busek, the autonomous optical navigation system,
the precision miniature reaction wheels, and high performance green
propellant and Honeywell's new Dependable Multiprocessor.\cite{Muelle_AsteroidProspector}

\subsection{Real-Time Geolocation with a Satellite Formation}

This study demonstrates that a group of two or three satellites in
LEO is able to accurately position a source of electromagnetic pulses
on the surface of the Earth. The position of the emitting source is
computed using time difference of arriving signals. The study also
talks about how this could be beneficial for other missions such as
the Mars rover or a busy constellation of positioning satellites.\cite{Leiter_Geoloc}

\subsection{NASA\textquoteright{}s GRAIL Spacecraft Formation Flight, End of Mission Results, and Small-Satellite Applications}

This mission, the Gravity Recovery and Interior Laboratory (GRAIL),
consists of two identical spacecraft designed to map the variations
of the gravitational field of the Moon. To do so, it is of great importance
that the spacecrafts measure the distance in between with high accuracy
and also orbit in the same orbital plane and height over the Moon's
surface.\cite{Edward_GRAIL}
