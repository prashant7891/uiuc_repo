\section{Planetary Science Missions}

\subsection{Mineral Mapping of Asteroids (Concept)}
The proposed mission overview is a single 6U CubeSat launched on a GEO satellite 
or Mars-bound mission as a secondary payload. There is a solar sail to reach near Earth asteroids. The proposed science objectives is to map surface composition of ~3 asteroids at 1-20 m spatial resolution.

Required instrumentation: ~spatial IFOV of 0.5 mrad, spatial sampling 0.5 m -10 m depending on the encounter range, Spectral sampling 10 nm, Imaging Spectrometer, 0.4 – 1.7 $\mu$m. Perhaps extend to 2.5 $\mu$m w/ HOT-BIRD or other advanced detector and achievable cooling. \href{http://kiss.caltech.edu/cosponsored/cubesat2012/presentations/staehle-interplanetary-cubesat-missions.pdf}{Link to presentation given by Robert Staehle}

\subsection{Solar system escape (Concept)}
The plan is to use a large area/ low mass spacecraft for high speed trajectory with low perihelion which would explore interplanetary environment, heliosheath and perhaps heliopause. It is also aimed to test communications, power, pointing and miniaturized instrument technologies.  \href{http://kiss.caltech.edu/cosponsored/cubesat2012/presentations/staehle-interplanetary-cubesat-missions.pdf}{Link to presentation given by Robert Staehle}

Required Instrumentation: Plasma, solar wind, Energetic particles \& cosmic rays, Magnetometer, Cameras to observe sail interaction with environment.


\subsection{Radio Quiet Lunar CubeSat (Concept)}
The aim is to assess radio quiet volume in shielded zone behind the Moon for future 21 cm cosmology missions. The proposed science mission is to find the usable volume behind the Moon for high sensitivity 21 cm cosmology observations which determines utility of lunar surface vs. orbiting missions.  \href{http://kiss.caltech.edu/cosponsored/cubesat2012/presentations/staehle-interplanetary-cubesat-missions.pdf}{Link to presentation given by Robert Staehle}

\subsection{Tracking Asteroids and Satellite Debris }

A formation of satellites would search for Earth-Approaching Asteroids
and potentially hazardous debris satellites using a small imaging
telescope \cite{Ref:Carroll}. One study suggests an orbit of 6000-40000
km \cite{Ref:Leitch}. 

Similarly, space debris of 1-10 cm is difficult to track using current
methods and can be very dangerous to things like solar arrays and
radiators \cite{Ref:Guerrero}. One study suggests the use of optical
and radar telescopes to track the small debris \cite{Ref:Tolkachev}.

