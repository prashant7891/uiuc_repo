\section{Our Mission Proposals}

\subsection{Division of Labor in Satellite}
Every year we throw away satellites because they become worn out. But usually its only one of the component that causes the problem. The other components work fine but is needlessly thrown away. This makes satillites as one of the causes of the space trash build up. The DARPA's Pheonix satellite re-servicing mission addresses these issues and tries to utilize functioning part of the satellites. But instead of trying clean up what has already happened (like  re-using parts of defunct satellites), here we propose a mission to reduce problems in the future.
The idea here is to apply division of labor. Instead of a one big satellite with all the components, a smaller satellites carrying one or two parts could fly an a constellation to represent a system as a whole. To list  benefits: \\

{\bf Expandable:} If more computing power is needed, a CubeSat carrying another CPU could join the constellation. \\
 
{\bf Customizable:} Camera carrying CubeSat could be replaced with a CubeSat with different type of sensor.
If there is free space in terms of data transfer or processing power, same or different company could send in just the sensor unit to collect data. \\

{\bf Reconfigurable:} Working parts could be re-used for different missions.\\

{\bf Green:} Creates less space waste.

\subsection{Satellite Advertising}
If CubeSats could carry reflective surface or an LED to make itself visible from earth, tightly controlled formation flying CubeSats could spell out a word or even form a display screen in great numbers. Indeed, it would be most likely only visible with the help of a telescope but just by the existance of space ad would have great impact on the media. Along with star viewing, it would become "the thing" to look at. Depending on visible areas at that time, different advertisements could be displayed.

\subsection{Orbiting Marker in Space}
Controlled formation flying satellites could serve as a marker for vision based navigations. Multiple orbiting markers on some other planet could allow better pose estimation than existing star tracker.

\subsection{Graveyard Orbit Transfer Service Formation Flying Proposal}
\subsubsection{Summary}
At the end of a satellite's service life, a satellite standard exists for the disposal of a spacecraft. For a geosynchronous spacecraft, it is typical to deposit the spacecraft in an orbit with a perigee altitude above 36,100 km \cite{standard}. This orbit is chosen due to the small $\Delta V$ required to move a spacecraft into the orbit (approximately 7.4 meters per second) and the tendency for the spacecraft to not return to an orbit used by operating spacecraft \cite {SMAD}. Recently however, only about one-third of GEO spacecraft that have reached the end of their life time have successfully achieved a so called ``graveyard orbit" \cite{ESA}. The proposed formation flying spacecraft system seeks to increase the percent of spacecraft that successfully maneuver to a ``graveyard orbit" in a cost effective manner. 

\subsubsection{Concept of Operations}
As a proposed disposal method consider a satellite in geosynchronous orbit that has just reached the end of its operational life time and cannot move to a graveyard orbit. A formation of CubeSat satellites (used for their extreme cost effectiveness) form together, dock with the dead spacecraft, and fire thrusters to move the dead spacecraft to a graveyard orbit. Because it is difficult to know ahead of time which spacecraft will successfully move themselves to a graveyard orbit and which will not, a formation flight is essential to create re-configurable solutions for the many spacecraft that may or may not fail. There will be multiple types of 1U, 2U, and 3U CubeSats. Some CubeSats will serve as attitude and position controllers for the formation once formed together. Others will serve as major thrusters to move the dead satellite out of the geosynchronous orbit. Finally, the remaining CubeSats will act as fuel tanks for the system.

\subsubsection{Proof of Concept CubeSat Mission}
As a proof of concept, a 6U CubeSat could be launched along with three 1U CubeSats. The 6U CubeSat will act as a dead satellite. One of the 1U satellites will serve as a thruster, another as a fuel tank, and the final one as a controller of the formation. The controller will work to form a solution out of the three 1U CubeSats. The re-configured spacecraft will then dock with the 6U spacecraft and move it to a different orbit.

\subsection{3D Printing Replacement Parts For Dying Satellites}

When satellites start to degrade or suffer from micrometeoriod impacts, replacement parts can revitalize a satellite that would otherwise need to be deorbited. Launching replacement parts can be expensive, in some cases more expensive than launching a new satellite. 3D printing new pieces for eligible satellites would reduce the upkeep costs of satellite constellations and potentially reduce the orbital debris.  \\

3D printing technology for space (by Made In Space) is slated to be tested in orbit in 2014 and has already succeeded in parabolic flight tests. Limitations may exist as to size and placement of replacement part on the satellite, so it may be necessary to study a set of satellites with common buses, like TDRSS or GPS to assess what parts are possible to replace. 

The satellite formation would have one or more satellites dedicated to printing and other satellites dedicated to removing and replacing the part on the broken satellite. The 3D printing could occur while the satellites travel to the orbit of the broken satellite. Formation flying would be necessary during the hand-off of the part and possibly also during the printing itself depending on the requirements of the printer. 

\subsection{Tracking Gamma Ray Bursts (GRBs)}
A formation could be used to source GRBs through precise triangulation. With accurate measurements of inter-satellite distance and GRB incident time, the satellites could potentially increase the accuracy of GRB detection and positioning. \cite{Ref:Dill}. This has since been proven unnecessary, because current satellites use gamma ray detectors and UV detectors together, with the UV detectors calculating the positioning. A formation of gamma ray detectors can still be useful to verify the measurements of current satellites like NASA's Fermi, or possibly to address gaps in coverage. 

The gamma ray detector formation could also look for gamma rays emitted by thunderstorms. Fermi has observed the phenomenon, but it could possibly do with more study, to help Fermi differentiate between the real gamma ray signals and the thunderstorm generated gamma rays.\cite{Ref:Fermi} \cite{Ref:Kitts}

\subsection{Tomography of the Ionosphere/Auroras}

Depletion in ionospheric plasma can disrupt signal transferrance,
and not much is known about the depletion zones. The formation would
study how the depletions change and propagate so that scientists can
create a model and further their understanding of the phenomenon.
The satellites would be in a 360 km orbit at an inclination of 52
degrees \cite{Ref:Krause}\cite{Ref:Bracikowski}. These measurements
can also help to show the interactions between the thermosphere and
ionosphere \cite{Ref:Blalthazor}. 

Currently scientists studying auroras have to collapse their 3D models
into two dimensions to verify them. The tomography created by the
formation would allow the scientists to verify their models in 3 dimensions,
much more accurately than the current process. 


\subsection{Test Inflatable De-Orbit Device }

The formation of satellites could be used to test an inflatable, such
as a de-orbiting device. Some of the satellites would not have inflatables
to determine the difference between the satellites with de-orbiter
and those without. The formation would not only test the inflatable
deployment, but also the efficacy against standard orbit degradation. 

\subsection{Variable-Range Solar Coronography}

Coronal mass ejections (CMEs) are a common phenomena in our Sun and
they can have a significant impact in our daily life. CMEs release
enormous amounts of plasma and energy into space inducing violent
solar winds. If this wind reaches our planet it can alter the Earth's
magnetosphere, cause remarkable aurora and malfunctioning in electrical
distribution lines, interfere in our communications and even damage
our satellites. For these reasons, the ability to predict and model
the influence of CMEs on the Earth is of great importance. Therefore,
several space missions are currently dealing with this issue, such
as:
\begin{enumerate}
\item Solar and Heliospheric Observatory (\href{http://www.nasa.gov/mission_pages/soho/index.html\#.UjUrV8YagWA}{SOHO}):
it is a monolithic spacecraft located at the first Lagrangian point
(L1) and constantly facing the Sun. Among the various instruments
that SOHO carries there are several that are dedicated exclusively
to study the solar corona:

\begin{enumerate}
\item Large Angle and Spectrometric Coronagraph (\href{http://lasco-www.nrl.navy.mil/}{LASCO})
has become one of the most relevant instruments in SOHO. It consists
of three different coronographs that focus their attention on a different
area around the solar corona. After all, the data and images collected
from this three coronographs in three different nested regions around
the Sun is blended to create a single image and data collection. In
the following table there are several characteristics of these coronographs:


\begin{table}[H]
\begin{centering}
\begin{tabular}{|c|c|c|c|c|c|}
\hline 
\multirow{2}{*}{\textbf{Coronagraph}} & \textbf{Field of View} & \textbf{Occulter} & \textbf{Spectral} & \textbf{Objective} & \textbf{Pixel }\tabularnewline
 & \textbf{($R_{sun})$} &  & \textbf{ Bands} & - & \textbf{Size}\tabularnewline
\hline 
\hline 
C1 & 1.1-3.0 & Internal & Fabry-Perot & Mirror & 5.6''\tabularnewline
\hline 
C2 & 1.5-6.0 & External & BroadBand & Lens & 11.4''\tabularnewline
\hline 
C3 & 3.7-30 & External & BroadBand & Lens & 56.0''\tabularnewline
\hline 
\end{tabular}
\par\end{centering}

\caption{LASCO Coronographs \cite{LASCO_Coronal_Imaging_Spectroscopy}}


\end{table}


\item Coronal Diagnostic Spectrometer (\href{http://solar.bnsc.rl.ac.uk/}{CDS})
performs spectrometry of the atoms and ions in the solar corona and
the transition region.
\item Ultra Violet Coronagraph Spectrometer (\href{http://www.cfa.harvard.edu/uvcs/}{UVCS})
measures the spectrum characteristics of the highly ionized plasma
in the solar corona, studying a region between 1.3 and 12 solar radii.
\end{enumerate}
\item Solar Terrestrial Relations Observatory (\href{http://www.nasa.gov/mission_pages/stereo/main/index.html\#.UjUnXsYagWA}{STEREO})
consists of two spacecraft that simultaneously take pictures of the
solar corona from different positions in their orbit. Having two spacecraft
looking at the same region has allowed this mission to create the
first stereoscopic views of CMEs and other solar measurements. 
\end{enumerate}
The coronagraphs implemented in these missions have two main limitations.
The first is that for a fixed distance between the occulter and the
imaging sensor, the instrument is only capable to provide images of
a finite region above the Sun's limb. Furthermore, if the instrument
is focused on the study of the closest layers above the Sun's limb,
the majority of the imaging element is shadowed by the occulting disk.
The second limitation is the aperture due to size restrictions. Consequently,
LASCO instrument includes three different coronographs with nested
fields of view to create a full picture of the desired region around
the Sun. Furthermore, LASCO is not able to provide stereoscopic images
of CMEs, whereas STEREO does provide stereoscopic images, as it consists
of two different spacecrafts, but only focuses on a certain region
of the Sun's corona.

\textcolor{black}{Formation Flying Satellites could mean a great leap
in the study of the solar corona and CMEs. A range-variable stereoscopic
coronagraph could be implemented with 4 Formation Flying Satellites.
Separating the occulter disk and the imaging element in two different
spacecrafts will allow heliophysicist to focus their instrument on
the desired area and even follow CMEs as they expand into space just
by varying the distance between the spacecrafts containing the occulter
and the imaging element. Moreover, adding a pair of occulter and imaging
spacecrafts the mission will be able to provide stereoscopic images
of CMEs. As we keep adding more spacecraft to the formation the mission
can either improve the stereoscopic data or enhance the field of view.
The following picture is a concept of the mission:}

\begin{figure}[!ht]
\begin{centering}
\includegraphics[scale=0.8]{jpl_logo.jpg}\caption{Range-variable stereoscopic coronagraph}
\par\end{centering}
\end{figure}

\textcolor{black}{To estimate the attitude requirements of this kind
of mission the following restrictions have been imposed. The occulter
has to maintain a position in which the imaging sensor is never exposed
to Sun's direct radiation. Therefore the occulter shadows a circle
equivalent to$1.05R_{sun}\text{\textpm}0.025R_{sun}$. To do so we
propose a simplified system of 2 Formation Flying Satellites with
3 DOF ($d,\theta,\delta)$ represented in the following picture:}


\begin{figure}[!ht]
\begin{centering}
\includegraphics[scale=0.75]{jpl_logo.jpg}\caption{DOF for two Formation Flying Satellite Coronagraph}
\end{centering}
\end{figure}


\textcolor{black}{In the following table there is an estimation of
the attitude requirements for two Flight Formation CubeSats, assuming
$\theta=1\text{º}$:}

\begin{table}[H]
\centering{}%
\begin{tabular}{|c|c|c|c|c|c|c|c|c|c|c|}
\hline 
\multicolumn{2}{|c|}{Radius of occulter (m)} & \multicolumn{3}{c|}{0.243} & \multicolumn{3}{c|}{0.487} & \multicolumn{3}{c|}{0.974}\tabularnewline
\hline 
\multicolumn{2}{|c|}{$d_{max}$(m)} & \multicolumn{3}{c|}{50} & \multicolumn{3}{c|}{100} & \multicolumn{3}{c|}{200}\tabularnewline
\hline 
\multicolumn{2}{|c|}{$\text{\ensuremath{\Delta}d}$(\%)} & $\text{\textpm}1$ & $\text{\textpm}0.1$ & $\text{\textpm}0.01$ & $\text{\textpm}1$ & $\text{\textpm}0.1$ & $\text{\textpm}0.01$ & $\text{\textpm}1$ & $\text{\textpm}0.1$ & $\text{\textpm}0.01$\tabularnewline
\hline 
\hline 
$\delta$(m) & $\Delta\delta$(\%) & \multicolumn{9}{c|}{$\Delta\theta$(º)}\tabularnewline
\hline 
\multirow{3}{*}{0.1} & $\text{\textpm}1$ & - & - & - & - & 0.0022 & 0.0032 & 0.0133 & 0.0135 & 0.0135\tabularnewline
\cline{2-11} 
 & $\text{\textpm}0.1$ & - & 0.0166 & 0.0194 & 0.0112 & 0.0207 & 0.0216 & 0.0225 & 0.0228 & 0.0228\tabularnewline
\cline{2-11} 
 & $\text{\textpm}0.01$ & - & 0.203 & 0.0231 & 0.0131 & 0.0225 & 0.0235 & 0.0234 & 0.0237 & 0.0237\tabularnewline
\hline 
\multirow{3}{*}{0.01} & $\text{\textpm}1$ & 0.0138 & 0.0191 & 0.0196 & 0.0138 & 0.0210 & 0.0217 & 0.0138 & 0.0219 & 0.0227\tabularnewline
\cline{2-11} 
 & $\text{\textpm}0.1$ & 0.0175 & 0.0228 & 0.0233 & 0.0157 & 0.0228 & 0.0235 & 0.0147 & 0.0228 & 0.0236\tabularnewline
\cline{2-11} 
 & $\text{\textpm}0.01$ & 0.0179 & 0.0232 & 0.0237 & 0.0158 & 0.0230 & 0.0237 & 0.0148 & 0.0229 & 0.0237\tabularnewline
\hline 
\multirow{3}{*}{0.001} & $\text{\textpm}1$ & 0.0138 & 0.0228 & 0.0237 & 0.0138 & 0.0228 & 0.0237 & 0.0138 & 0.0228 & 0.0237\tabularnewline
\cline{2-11} 
 & $\text{\textpm}0.1$ & 0.0138 & 0.0228 & 0.0237 & 0.0138 & 0.0228 & 0.0237 & 0.0138 & 0.0228 & 0.0237\tabularnewline
\cline{2-11} 
 & $\text{\textpm}0.01$ & 0.0139 & 0.0228 & 0.0237 & 0.0138 & 0.0228 & 0.0237 & 0.0138 & 0.0228 & 0.0237\tabularnewline
\hline 
\end{tabular}\caption{Estimated requirements for two Formation Flight Satellite Coronagraph}
\end{table}


Finally, the mission requirements obtained are achievable with current
sensors and actuators available on the market. The proposed mission
is feasible and could lead to a great improvement in the flexibility
and data collection of heliophysics missions.


\subsection{CubeSat landing swarm to track and model asteroid belts}

An example of the current interest in the asteroid belt among the
scientific community is NASA mission \href{http://dawn.jpl.nasa.gov/mission/journal.asp}{Dawn}.
This mission has already placed a spacecraft orbiting around the gigantic
asteroid Vesta and now heading towards Ceres, where it is expected
to unveil some of the secrets of this body. Ceres is thought to be
an icy dwarf planet capable to give a reasonable explanation about
the origin of water in Mars and Earth. Therefore, Ceres and some of
the asteroids in this belt are quite interesting from an astrobiological
point of view as they contain ice water. In addition, some asteroids
have important amounts of metals and carbon compounds. Scientist believe
that this region holds some important clues about the origin and evolution
of the Solar System. Nevertheless the only outstanding mission to
the asteroid belt indicates that there is still much work to answer
all the questions scientist ask now. For that reason, a swarm of CubeSats
is a promising way to continue the exploration of the asteroid belt
in an efficient and flexible way.

A swarm of CubeSats capable to land on several asteroids has many
profitable applications although initially Flight Formation will not
be needed the mission could be notably improved with the aided help
of cooperating spacecrafts:
\begin{enumerate}
\item In an initial approach the CubeSats could provide accurate imaging
and mapping of a wide variety of asteroids. After landing on the surface
of the asteroid samples of them could be taken in order to study composition
and origin.
\item The swarm of CubeSats spread all over the asteroid belt could be used
to easily track smaller asteroids, improve the computational models
of the asteroid belt and precisely estimate collisions or orbital
variations of different asteroids.
\item This small satellites could be used to improve communications with
more distant spacecraft and even create a network all around the asteroid
belt. This will allow easy communications with spacecrafts even though
the Sun is in the way between the Earth and the spacecraft and therefore
blocks the incoming and outgoing signals.
\item There has been much speculation about the industrialization of the
asteroid belt and how this could supply materials and energy to missions
in Mars or beyond. The swarm of CubeSats could set the foundation
of this movement. Stackable CubeSats can lead to machine construction
and consequently to the asteroid belt industrialization.
\end{enumerate}

\subsection{Sun Energy Collector}

Harvesting energy has been and still being one of the critical aspects
of a space mission. Some examples of this problem are: 
\begin{enumerate}
\item The Mars Rover Opportunity, unable to perform its experiments on the
slopes of mountains and sand dunes where the sun does not shine during
the day. Hence, the mars rover has experienced important limitations
in its operations due to energy restrictions.
\item Spacecrafts traveling to the outer Solar System or even out of the
Solar System require important amounts of energy for their propulsion
systems in order to gain the needed velocity. Furthermore, the moons
of Jupiter and Saturn, Europa and Enceladus respectively, are thought
to hide liquid water beneath their frozen surfaces and therefore have
been classified as high priority for NASA and ESA. Consequently, there
is an obvious need for cheaper sources of energy for interplanetary
missions.
\item Satellites charge their batteries while exposed to the Sun's radiation.
Then, when they enter the night-side of a planet for a long period
of time, sometimes are forced to perform their science with low-power
modes in order to save energy.
\end{enumerate}
A group of Formation Flying Satellites with deployable mirrors could
be the solution for this scarcity of energy. The Sun is a constant
and powerful source of energy in space. Accordingly, spacecrafts and
ground systems could benefit more from the Sun's energy than they
are doing now. An orientable array of mirrors could be implemented
in several spacecrafts to provide energy to a wide range of missions:
\begin{enumerate}
\item Satellites in desperate need of energy could be rescued from thousands
of kilometers away just by re-orienting the Formation Flying Spacecrafts.
Or even better, critical satellites could be able to perform all their
experiments during long periods in the dark side of a planet. 
\item Vehicles and rovers in other planets or moons will have the ability
to continue their research during the night and after long periods
in shaded areas. For this, placing a group of Formation Flying Satellites
in orbit around the corresponding planet or moon will enable to point
a beam of light to those vehicles that require energy.
\item A beam of photons could be carefully pointed toward a solar sail to
increase solar pressure and help the classical propulsion systems
take the spacecraft to the outer solar system. By this means, space
agencies would be able to save millions of dollars in putting large
quantities of combustible into space.
\item A group of Formation Flying Satellites could be used as a test-bed
to determine how light can be used to deviate a small asteroid from
its trajectory.
\end{enumerate}