\section{Earth Science Missions}

\subsection{Cubesat investigating Atmospheric Density Response to Extreme driving (CADRE)}
This project's  main instrument is a 3-Unit (3U) CubeSat which is named as Cubesat investigating Atmospheric Density Response to Extreme driving (CADRE). The major issues addressed are related to ion-neutral coupling, which includes neutral wind morphology and dynamics which are very important for understanding how the thermosphere reacts to energy input and the role this plays in magnetosphere-ionosphere coupling. This is currently being carried out by Aaron Ridley at University of Michigan. \href{http://nsf.gov/awardsearch/showAward?AWD_ID=1042815&HistoricalAwards=false}{The link to the NSF award.}

\subsection{Collaborative Research: CubeSat--Composition Variations in the Exosphere, Thermosphere, and Topside Ionosphere (EXOCUBE)}

This project measures the densities of all significant neutral and ionized species in the upper atmosphere on a global scale. It is a 3U CubeSat. The main objective of this project is to provide the first in-situ global neutral density data in more than 25 years, which includes using the mass spectrometer technique to directly measure Hydrogen densities. This missions also provides observational constraints for physical models of the upper atmosphere. Also, newly developed experimental techniques which are used to obtain neutral and ionized composition and densities from radar and optical observations can be tested and validated using the measurements from this mission. This is currently being done by John Noto at Scientific Solutions Incorporated. \href{http://nsf.gov/awardsearch/showAward?AWD_ID=1042837&HistoricalAwards=false}{The link to the NSF award.}



\subsection{Ionosphere Monitoring}

\subsubsection{Predicting Earthquakes through Ionosphere Monitoring}

Early earthquake detection could help the third of a population that
is affected by earthquakes. Precursors to the earthquake can be detected
through variations in the ionosphere. The proposed satellite will
have a RAS topside sounder and be used to monitor the areas of the
Earth that have high seismic activity to predict earthquakes and prove
that the prediction method is valid and accurate. The data could also
help to improve GPS navigation models and study the reaction of the
ionosphere to magnetic events. The formation may help to mitigate
measurement errors.\cite{Ref:Jason} Something similar is scheduled
to be launched by the Chinese in 2014, though not a formation\cite{Ref:Earthquake}. 


\subsubsection{Studying the Reaction of the Ionosphere to Storms }

Magnetic storms can cause changes and bulges in the ionosphere. Understanding
these bulges can help us to understand the magnetic storm that causes
them as well. The DICE mission has studied this using 2 satellites
equipped with langmuir and electric field probes to measure the plasma
density and field strength. \cite{Ref:Crowley,Ref:Fish}

The ionospheric measurements can also track thunderstorms and cyclones
if positioned near the oxygen absorption line (\textasciitilde{}500
km) at a near-equatorial orbit using a microwave spectrometer. \cite{Ref:Blackwell}


\subsubsection{Monitoring Atmospheric Plasma Depletion to Predict Outages in GPS and Communications}

Depletion in ionospheric plasma can disrupt signal transferrance,
and not much is known about the depletion zones. The formation would
study how the depletions change and propagate so that scientists can
create a model and further their understanding of the phenomenon.
The satellites would be in 360 km orbits at an inclination of 52 degrees
\cite{Ref:Krause}\cite{Ref:Bracikowski}. These measurements can
also help to show the interactions between the thermosphere and ionosphere
\cite{Ref:Blalthazor}.

\subsection{Earth Imaging for Science Applications in Emerging Countries}

A satellite imaging cluster would give less advanced parts of the
world access to scientific data on things like resource consumption,
pollution, and climate. The formation could do the imaging with each
satellite operating a different camera type. The placement can range
from 400-720 km, but the effects of the high drag environment make
control much more difficult\cite{Ref:Harrison} \cite{Ref:Atir} \cite{Ref:Herscovitz}\cite{Ref:Kalman}\cite{Ref:Kim}.

The Nigerian government is launching a satellite of this type to monitor
environmental issues within the country, provide high volume mapping
data, and highly accurate image targeting and geolocation\cite{Ref:Curiel}. 

\subsection{Observing Gamma Rays Emitted by Thunderstorms}

The satellites will look for gamma rays emitted by thunderstorms in
visual and radio frequencies. NASA's Fermi telescope has observed
the phenomenon, but it could possibly do with more study and happen
in conjunction with the GRB monitoring.\cite{Ref:Fermi} \cite{Ref:Kitts}

\subsection{Space-Based Ocean Monitoring}

The health of Earth's bodies of water can be monitored through multi-spectral imaging with high spatial and temporal resolution. This will help scientists to better understand the effects of tides on ocean color, as well as the evolution of ecosystems. One study proposes using 115 nanosats for global coverage, but this is extremely ambitious, the formation would be much much smaller. \cite{Ref:Lowe}. The formation may not be able to add to the science.

\subsection{Testing Satellite Tether Deployment and Operations}

Satellite tethers can be used to create artificial gravity to aide
in long term human missions by tethering a crew module to an object
of equal mass and rotating the system. These systems need to be tested before they can be used for such operations. \cite{Ref:Carlson}\cite{Ref:Mazzoleni}

\subsection{Completing the Map of the Earth's Electric Field}

This project would use a system of small satellites to observe the
Earth's electric field with radar measurements. A proposed project
uses a constellation of 48 satellites. \cite{Ref:Redd}. Since it
is more appropriate for a constellation than for a formation, it is
not a great candidate for our project unless we wanted to create a
3D map. It would appear that other systems already make these measurements, but there are still areas of poor coverage that could be addressed \cite{Ref:ElecMap}. 

\subsection{Raman Spectroscopy to investigate the atmosphere}

Create a more in-depth model of the upper atmosphere using Raman Spectroscopy from multiple sources in a formation to achieve a 3D (or at least more comprehensive) map of planetary mineral and chemical abundances. The same technique could be used to study other planets as well \cite{Ref:Cantrell}. The main obstacle here would be finding a detector array that would fit in our size constraints. 

\subsection{Formation Flying to Sample Volume of Magnetosphere}

Use the formation of cubesats to create a more detailed 3D model of
the magnetosphere, adding detailed dynamic measurements. This may
not be necessary because of the twin satellites launched by NASA in
Fall 2012. Also several earlier missions acheived similar results,
not using formations but I doubt the information is still needed.\cite{Ref:Mag3D} In one proposed mission, a master satellite ejects several picosats which take 2-axis magnetometer data and relay it to the master. Relative position and attitude control are not necessary as long as the positions and attitudes can be discerned.\cite{Ref:Clarke}