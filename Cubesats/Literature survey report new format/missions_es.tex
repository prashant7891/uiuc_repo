\section{Earth Science Missions}

\subsection{CubeSat investigating Atmospheric Density Response to Extreme driving (CADRE)}
This project's  main instrument is a 3-Unit (3U) CubeSat which is named as CubeSat investigating Atmospheric Density Response to Extreme driving (CADRE). The major issue investigated is related to ion-neutral coupling, which includes neutral wind morphology and dynamics which are very important for understanding how the thermosphere reacts to energy input and the role this plays in magnetosphere-ionosphere coupling. This is currently being carried out by Aaron Ridley at University of Michigan. \href{http://nsf.gov/awardsearch/showAward?AWD_ID=1042815&HistoricalAwards=false}{The link to the NSF award.}

\subsection{Collaborative Research: CubeSat--Composition Variations in the Exosphere, Thermosphere, and Topside Ionosphere (EXOCUBE)}

This project measures the densities of all significant neutral and ionized species in the upper atmosphere on a global scale. It is a 3U CubeSat. The main objective of this project is to provide the first in-situ global neutral density data in more than 25 years, which includes using the mass spectrometer technique to directly measure hydrogen densities. This missions also provides observational constraints for physical models of the upper atmosphere. Also, newly developed experimental techniques which are used to obtain neutral and ionized composition and densities from radar and optical observations can be tested and validated using the measurements from this mission. This is currently being done by John Noto at Scientific Solutions Incorporated. \href{http://nsf.gov/awardsearch/showAward?AWD_ID=1042837&HistoricalAwards=false}{The link to the NSF award.}



\subsection{Ionosphere Monitoring (Concept)}

\subsubsection{Predicting Earthquakes through Ionosphere Monitoring}

Early earthquake detection could help one third of the world's population. Precursors to the earthquake can be detected through variations in the ionosphere. The proposed satellite will have a RAS topside sounder and be used to monitor the areas of the Earth that have high seismic activity to predict earthquakes and prove that the prediction method is valid and accurate. The data could also help to improve GPS navigation models and study the reaction of the ionosphere to magnetic events. The formation may help to mitigate measurement errors \cite{Ref:Jason}. A similar mission is scheduled to be launched by the Chinese Space Agency in 2014, though that one is not a formation\cite{Ref:Earthquake}. 


\subsubsection{Studying the Reaction of the Ionosphere to Storms }

Magnetic storms can cause changes and bulges in the ionosphere. Understanding these bulges can help us to understand the magnetic storm that causes them as well. The DICE mission has studied this using 2 satellites equipped with langmuir and electric field probes to measure the plasma density and field strength \cite{Ref:Crowley,Ref:Fish}. The ionospheric measurements can also track thunderstorms and cyclones if positioned near the oxygen absorption line (\textasciitilde{}500 km) at a near-equatorial orbit using a microwave spectrometer. \cite{Ref:Blackwell}


\subsubsection{Monitoring Atmospheric Plasma Depletion to Predict Outages in GPS and Communications}

Depletion in ionospheric plasma can disrupt signal transferrance, and little is known about the depletion zones. The formation would study how the depletions change and propagate so that scientists can create a model and further their understanding of the phenomenon. The satellites would be in 360 km orbits at an inclination of 52 degrees \cite{Ref:Krause}\cite{Ref:Bracikowski}. These measurements can also help to show the interactions between the thermosphere and ionosphere \cite{Ref:Blalthazor}.

\subsection{Earth Imaging for Science Applications in Emerging Countries}

A satellite imaging cluster would give less advanced parts of the world access to scientific data on things like resource consumption, pollution, and climate. The formation could do the imaging with each satellite operating a different camera type. The placement can range from 400-720 km, but the effects of the high drag environment make control much more difficult\cite{Ref:Harrison} \cite{Ref:Atir} \cite{Ref:Herscovitz}\cite{Ref:Kalman}\cite{Ref:Kim}. For example, the Nigerian government is launching a satellite of this type to monitor environmental issues within the country, provide high volume mapping data, and highly accurate image targeting and geolocation\cite{Ref:Curiel}. 

\subsection{Observing Gamma Rays Emitted by Thunderstorms}

The satellites will look for gamma rays emitted by thunderstorms in visual and radio frequencies. NASA's Fermi telescope has observed the phenomenon, but some more analysis of this phenomenon can help NASA and this analysis happen in conjunction with the GRB monitoring.\cite{Ref:Fermi} \cite{Ref:Kitts}

\subsection{Space-Based Ocean Monitoring}

The health of Earth's bodies of water can be monitored through multi-spectral imaging with high spatial and temporal resolution. This will help scientists to better understand the effects of tides on ocean color, as well as the evolution of ecosystems. One study proposes using 115 nanosats for global coverage, but this is extremely ambitious, the formation would be much much smaller. \cite{Ref:Lowe}. %The formation may not be able to add to the science.

\subsection{Completing the Map of the Earth's Electric Field}

This project would use a system of small satellites to observe the Earth's electric field with radar measurements. A proposed project uses a constellation of 48 satellites. \cite{Ref:Redd}. However, unless we want to create a 3D map this mission is not appropriate as it is designed for a constellation rather than for a formation. It would appear that other systems already make these measurements, but there are still areas of poor coverage that could be addressed \cite{Ref:ElecMap}. 

\subsection{Raman Spectroscopy to investigate the atmosphere}

Create a more in-depth model of the upper atmosphere using Raman Spectroscopy from multiple sources in a formation to achieve a 3D (or at least more comprehensive) map of planetary mineral and chemical abundances. The same technique could be used to study other planets as well \cite{Ref:Cantrell}. The main obstacle in this mission would be finding a detector array that would fit in our size constraints. 

\subsection{Formation Flying to Sample Volume of Magnetosphere}

Use a formation of CubeSats to create a more detailed 3D model of the magnetosphere, adding detailed dynamic measurements. However, NASA launched twin satellites in Fall 2012 which perform a similar mission. Also several earlier missions have achieved similar results without using formations\cite{Ref:Mag3D}. Hence the need for a formation is not established here. In one proposed mission, a master satellite ejects several picosats which take 2-axis magnetometer data and relay it to the master. Relative position and attitude control are not necessary as long as the positions and attitudes can be discerned.\cite{Ref:Clarke}

\subsection{Satisfying Australia's Future Need for Multi-Spectral Imagery}

Geosciences Australia studied the need for satellite images of the Australian geography. As a result, the company created a document titled ``Continuity of Earth Observation Data for Australia - Operational Requirements to 2015 for Lands, Coasts and Oceans.'' which contains their conclusions. The most relevant idea is that there is a need to image the Australian landmass daily which will not be covered with the launch of the newest public satellites. To satisfy the imaging requirements of Australia the paper proposes a 6U CubeSat with commercial software to help the bigger satellites and obtain medium spatial resolution and high temporal resolution. \cite{Stepan_ConstellationImagery}

\subsection{Small Satellite Constellations for Earth Geodesy and Aeronomy(Concept)}

Drag-free nanosatellites consist of an external shielding that contains a free-floating mass. By measuring the movement of the free-floating mass with respect to the shielding, this satellites are able to precisely compute the drag and therefore determine their orbital position with more accuracy. Consequently, this kind of satellites can  improve the accuracy and sensitivity of the aeronomy and geodesy measurements, specially if they cooperate with other satellites to make those measurements. This characteristic can be used to map the Earth's mass distribution
with high precision, compute high-order harmonics or even try to detect gravitational waves.\cite{Conklin_GeoAero_Constellation}

\subsection{A 6U CubeSat Constellation for Atmospheric Temperature and Humidity Sounding}

This paper describes the development and implementation of a 118 GHz temperature sensor and a 183 GHz humidity sensor which are suitable for a 6U CubeSat. In addition, the appropriate design of a 10 cm antenna provides a sufficient footprint of approximately 25 km. The paper takes advantage of the technology developed for the High Altitude Microwave Scanning Radiometer (HAMSR) by JPL. \cite{Sharmila_ContellationTempHumidity}

\subsection{Simultaneous Multi-Point Space Weather Measurements using the Low Cost EDSN CubeSat Constellation}

The Edison Demonstration of Smallsat Networks (EDSN) mission consists of eight CubeSats that will be launched to a low-Earth orbit. These CubeSats are intended to monitor spatial and temporal variations in the radiation levels by cooperating in loose formation at approximately 500 km above the Earth's surface. This will allow a better understanding of the space weather that could lead to an improvement of the current space weather models. EDSN mission will carry the Energetic Particle Integrating Space Environment Monitor (EPISEM) that will determine the radiation environment by taking measurements over a dispersed area of energetic charged particles. The expected launch date for this mission is late 2013 as a secondary payload on a DoD mission.\cite{Gunderson_MultiptMeasureConstellation}\cite{Yost_MultipointMeasurement}

\subsection{Design of Nano-satellite Cluster Formations for Bi-Directional Reflectance Distribution Function (BRDF) Estimations}

The Bidirectional Reflectance Distribution Function (BRDF) evaluates the variation of reflectivity in terms of direction and spectrum over the surface of the Earth. This function is essential to determine various parameters such as albedo which is currently estimated using a wide 3D angular range of illumination and direction sensors in the visible and near infrared wavelengths. This paper proposes a nano-satellite cluster to improve the computation of BRDF. The project is presented as an additional help for current BRDF systems.\cite{Nag_BRDF}

\subsection{Design and Analysis of a Nanosatellite Platform for Orbital Debris Mitigation through Launch of Space Tether in Low Earth Orbits}

This paper proposes a nano-satellite platform, with commercial of the shelf components, to tether space debris by instantly solving Lambert's problem. The nano-satellite platform will contain a tether launching system with two Terminator Tape tethers built by the company Tethers Unlimited. Space debris will be spotted by the nano-satellite in cooperation with the ground stations. Once the space debris is targeted, the nano-satellite will predict the trajectory of the debris and perform the proper launching maneuvers, determined from by solving the Lambert's problem, for the Terminator Tape to grab the space debris.\cite{Asundi_SpaceDebris} 

\subsection{Collapsible Space Telescope (CST) for Nanosatellite Imaging and Observation}

A low-cost collapsible Cassegrain telescope that can fit in a 4U portion of a 6U CubeSat is being developed by NASA Ames Research Center. The intention of this mission is to obtain high-resolution imaging of Earth and space observations when coupled with an appropriate imaging sensor. The implementation of this kind of telescope in a small satellite sets the path to swarm missions with distributed apertures.\cite{Elwood_CollapsibleTelescope}

\subsection{FTS CubeSat Constellation Providing 3D Winds}

This paper describes the technological and scientific developments required for the implementation of a Fourier Transform Spectrometer (FTS) instrument. The FTS is intended to fit in a 6U CubeSat and would allow more precise tropospheric wind forecasts from space. To achieve that goal, the mission requires flight formation so the CubeSats know the inter-satellite distance and cooperate to provide measurements in different layers of the atmosphere. The measurements taken then can be overlapped to create 3D profiles of the atmospheric wind. This would mean a great improvement in current weather models and will allow longer-term weather forecasts. \cite{Wloszek_3Dwinds}