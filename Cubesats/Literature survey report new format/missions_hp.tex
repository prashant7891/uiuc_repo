\section{Heliophysical Missions}

\subsection{Colorado Student Space Weather Experiment}

This is a three-year multi-disciplinary team effort and the aim is to and operate a CubeSat. This 3U Cubesat carries an energetic particle sensor which will address fundamental space weather science questions regarding topics like relationship between solar flares, energetic particles and geomagnetic storms in the near Earth space environment. The particle instrument is the Relativistic Electron and Proton Telescope integrated little experiment (REPTile). REPTile is designed to measure directional differential flux of energetic protons, 10-40 MeV, and electrons, 0.5 to $\textgreater$ 3 MeV.  The major science objectives of this project are to investigate the relationships between solar energetic particles, flares, and coronal mass ejections, and also to characterize the variations of the Earth's radiation belt electrons. This is currently being carried out by Xinlin Li at University of Colorado at Boulder. \href{http://nsf.gov/awardsearch/showAward?AWD_ID=1042815&HistoricalAwards=false}{The link to the NSF award.}

\subsection{Solar Polar Imager CubeSat Constellation (Concept)}
This mission is a concept of 6 spacecrafts in a constellation with highly inclined orbits. They would be in an out-of-ecliptic vertical orbit. It will use solar sail to reach high inclination. The proposed science missions are helioseismology and measuring magnetic fields of polar regions, polar view of corona, coronal mass ejections, solar radiance and to link high latitude solar wind and energetic particles to coronal sources.  \href{http://kiss.caltech.edu/cosponsored/cubesat2012/presentations/staehle-interplanetary-cubesat-missions.pdf}{Link to presentation given by Robert Staehle}

The required instrumentation of the 6  satellites are:
\begin{itemize}
\item S/C1: Plasma + Mag Field 
\item S/C2: Energetic Particles + Mag Field 
\item S/C3: Cosmic Rays, 
\item S/C4: Magnetograph/Doppler Imager 
\item S/C5: EUV Imager 
\item S/C6: Coronagraph 
\end{itemize}


\subsection{Earth-Sun Sunward-of-L1 Solar Monitor (Concept)}
The aim of this concept is to measure strong coronal mass ejections or other space weather from Sunward-of-L1 position to provide additional warning time to Earth. The science objective is to obtain plasma and magnetometer readings of solar wind from sunward-of-L1 position and to compare with L1 values from ACE or follow-on. \href{http://kiss.caltech.edu/cosponsored/cubesat2012/presentations/staehle-interplanetary-cubesat-missions.pdf}{Link to presentation given by Robert Staehle}
