%%% LyX 2.0.6 created this file.  For more info, see http://www.lyx.org/.
%%% Do not edit unless you really know what you are doing.
%\documentclass[english]{article}
%\usepackage[LGR,T1]{fontenc}
%\usepackage[utf8]{inputenc}
%\usepackage{textcomp}
%
%\makeatletter
%
%%%%%%%%%%%%%%%%%%%%%%%%%%%%%%% LyX specific LaTeX commands.
%\DeclareRobustCommand{\greektext}{%
%  \fontencoding{LGR}\selectfont\def\encodingdefault{LGR}}
%\DeclareRobustCommand{\textgreek}[1]{\leavevmode{\greektext #1}}
%\DeclareFontEncoding{LGR}{}{}
%\DeclareTextSymbol{\~}{LGR}{126}
%
%\makeatother
%
%\usepackage{babel}
%\begin{document}
%
%\title{Formation Flying Missions }
%
%
%\title{Even Years from Small Satellites Conference}
%
%
%\author{Rebecca Foust}
%
%\maketitle

\subsection{Predicting Earthquakes through Ionosphere Monitoring}


\subsubsection{Mission Concept}

Fluctuations in the ionosphere occur hours or days before large earthquakes
{[}http://solar-center.stanford.edu/SID/educators/earthquakes.html{]}.
The formation may help to mitigate measurement errors. Ionosphere
measurement is also a target for other missions, if one of them is
adopted, this could be a secondary objective. Something similar is
scheduled to be launched by the Chinese in 2014, though not a formation. 


\subsubsection{Abstract}


\subparagraph{Earthquake Forecast Science Research with a Small Satellite (2002-Session
9)\cite{Ref:Jason}}

Reliable, repeatable Earthquake forecast is a subject surrounded by
controversy and scepticism. What is clear, is that reliable forecast
could be the single most effective tool for earthquake disaster management.
Roughly a third of the world’s population live in areas that are at
risk and, every year since the beginning of the twentieth century
earthquakes have caused an average of 20,000 deaths. The economic
loss in the 1995 Kobe, Japan earthquake was greater than US\$100 billion
. Substantial progress has been made on the development of methods
for earthquake hazard analysis on a timescale of a few decades. However,
the forecast of specific earthquakes on timescales of a few years
to a few days is a difficult problem. It has been proposed that satellites
and ground-based facilities may detect earthquake precursors in the
ionosphere a few hours or days before the main shock. This hypothesis
is now backed by a physical model, derived by the Russian Academy
of Sciences from statist ical studies, and an understanding of the
main morphological features of seismoionospheric precursors - which
allows them to be separated from background ionospheric variability.
The main problems now are lack of regular global data and limited
funding for what is considered to be financially risky research. Low
-cost, small satellites offer a solution to these problems. A 100
kg class SSTL enhanced microsatellite, carrying a RAS topside sounder
and complimentary payload, will be used to make regular measurements
over seismically active zones around the globe. The low cost of the
spacecraft offers a financially low -risk approach to the next step
in this invaluable research. The spacecraft will make ionospheric
measurements for systematic research into the proposed precursors.
The aims will be to confirm or refute the hypothesis; define their
reliability and reproducibility; and enable further scientific understanding
of their mechanisms. In addition, forecasting of the magnitude of
the events, as well as an indication of the seismic centre may also
be possible. These mission data should also lead to improved knowledge
of the physics of earthquakes, improved accuracy for GPS-based navigation
models, and could be used to study the reaction of the global ionosphere
during magnetic storms and other solar-terrestrial events. The paper
presents an overview of the scientific basis, goals, and proposed
platform for this research mission.


\subparagraph{Dynamic Ionosphere Cubesat Experiment (DICE) (2010- Session 3) (2012-
Session 11) \cite{Ref:Crowley,Ref:Fish}}

The Dynamic Ionosphere Cubesat Experiment (DICE) mission has been
selected for flight under the NSF \textquotedbl{}CubeSat-based Science
Mission for Space Weather and Atmospheric Research\textquotedbl{}
program. The mission has three scientific objectives: (1) Investigate
the physical processes responsible for formation of the midlatitude
ionospheric Storm Enhanced Density (SED) bulge in the noon to post-noon
sector during magnetic storms; (2) Investigate the physical processes
responsible for the formation of the SED plume at the base of the
SED bulge and the transport of the high density SED plume across the
magnetic pole; (3) Investigate the relationship between penetration
electric fields and the formation and evolution of SED. The mission
consists of two identical Cubesats launched simultaneously. Each satellite
carries a fixed-bias DC Langmuir Probe (DCP) to measure in-situ ionospheric
plasma densities, and an Electric Field Probe (EFP) to measure DC
and AC electric fields. These measurements will permit accurate identification
of storm-time features such as the SED bulge and plume, together with
simultaneous co-located electric field measurements which have previously
been missing. The mission team combines expertise from ASTRA, Utah
State University/Space Dynamics Laboratory (USU/SDL), Embry-Riddle
Aeronautical University and Clemson University.


\subparagraph{Conducting Science with a CubeSat: The Colorado Student Space Weather
Experiment (2010- Session 12)\cite{Ref:Palo}}

Energetic particles, electrons and protons either directly associated
with solar flares or trapped in the terrestrial radiation belt, have
a profound space weather impact. A 3U CubeSat mission with a single
instrument, the Relativistic Electron and Proton Telescope integrated
little experiment (REPTile), has been selected by the National Science
Foundation to address fundamental questions pertaining to the relationship
between solar flares and energetic particles. These questions include
the acceleration and loss mechanisms of outer radiation belt electrons.
The Colorado Student Space Weather Experiment operating in a highly
inclined low earth orbit, will measure differential fluxes of relativistic
electrons in the energy range of 0.5-2.9 MeV and protons in 10-40
MeV. This project is a collaborative effort between the Laboratory
for Atmospheric and Space Physics and the Department of Aerospace
Engineering Sciences at the University of Colorado, which includes
the integration of students, faculty, and professional engineers.


\subparagraph{Nanosatellites for Earth Environmental Monitoring: The MicroMAS Project
(2012- Session 1)\cite{Ref:Blackwell}}

The Micro-sized Microwave Atmospheric Satellite (MicroMAS) is a dual-spinning
3U CubeSat equipped with a passive microwave spectrometer that observes
nine channels near the 118.75-GHz oxygen absorption line. The focus
of this MicroMAS mission (hereafter, MicroMAS-1) is to observe convective
thunderstorms, tropical cyclones, and hurricanes from a near-equatorial
orbit. The MicroMAS-1 flight unit is currently being developed by
MIT Lincoln Laboratory, the MIT Space Systems Laboratory, and the
MIT Department of Earth and Planetary Sciences for a 2014 launch to
be provided by the NASA CubeSat Launch Initiative program. As a low
cost platform, MicroMAS offers the potential to deploy multiple satellites
than can provide near-continuous views of severe weather. The existing
architecture of few, high-cost platforms, infrequently view the same
earth area which can miss rapid changes in the strength and direction
of evolving storms thus degrading forecast accuracy. The 3U CubeSat
has dimensions of 10 x 10 x 34.05 cm3 and a mass of approximately
4 kg. The payload is housed in the “lower” 1U of the dualspinning
3U CubeSat, and is mechanically rotated approximately once per second
as the spacecraft orbits the Earth. The resulting cross-track scanned
beam has a FWHM beam width of 2.4º, and has an approximately 20-km
diameter footprint at nadir incidence from a nominal altitude of 500
km. Radiometric calibration is carried out using observations of cold
space, the Earth's limb, and an internal noise diode that is weakly
coupled through the RF front-end electronics. In addition to the dual-spinning
CubeSat, a key technology development is the ultra-compact intermediate
frequency processor (IFP) module for channelization, detection, and
analog-to-digital conversion. The payload antenna system and RF front-end
electronics are highly integrated, miniaturized, and optimized for
low-power operation. To support the spinning radiometer payload, the
structures subsystem incorporates a brushless DC zerocogging motor,
an optical encoder and disk, a slip ring, and a motor controller.
The attitude determination and control system (ADCS) utilizes reaction
wheels, magnetorquers, Earth horizon sensors, peak power tracking,
a magnetometer, and a gyroscope. The communications system operates
at S-band using the Open System of Agile Ground Stations (OSAGS) with
a 2.025—2.120 GHz uplink and 2.200—2.300 GHz downlink at 230 kbps.
MicroMAS-1 uses a Pumpkin CubeSat Motherboard with a Microchip PIC24
microcontroller as the flight computer running Pumpkin’s Salvo Real
Time Operating System. Thermal management includes monitoring with
thermistors, heating, and passive cooling. Power is generated using
four double-sided deployable 3U solar panels and one 2U bodymounted
panel with UTJ cells and an electrical power system (EPS) with 30
W-hr lithium polymer batteries from Clyde Space. Tests with the MicroMAS-1
Engineering Design Model (EDM) have resulted in modifications to the
spinning assembly, stack and ADCS system and have informed the development
of the flight model subsystems.


\subsection{Earth Imaging for Science Applications in Emerging Countries}


\subsubsection{Mission Concept}

A satellite imaging cluster would give less advanced parts of the
world access to scientific data on things like resource consumption,
pollution, and climate. The formation could do the imaging with each
satellite operating a different camera type.


\subsubsection{Abstract}


\subparagraph{NanoObservatory: Earth Imaging for Everyone (2002- Session 3)\cite{Ref:Harrison}}

Earth imaging has traditionally been the domain of large governments
and expensive satellites. Progress in earth imaging satellite technology
has focused on driving down image resolutions, decreasing re-visit
times, and expanding spectral coverage. While imaging capability has
increased, costs have grown exponentially, pricing many would-be science
applications out of the market. Many scientific groups, especially
those in countries with emerging economies, have a compelling need
for earth imaging to monitor the use of natural resources, measure
changes in climate, quantify and track pollution, and assist in natural
disaster warning and recovery—without the high cost of a dedicated
system or the restrictions imposed by sharing data from another country’s
system. NanoObservatory, shown in Figure 1, is a low-cost solution
for users wanting basic earth imaging for science applications. It
provides multi-spectral imaging (red, green, blue) with a ground sampling
distance (GSD) of 10m, and can be customized to image in other spectral
ranges as necessary. The satellite resides in a 600km circular orbit
between 0º and 38º inclination. From this vantage point, the satellite
images a 50km x 50km area and can store consecutive images to create
a seamless view. The satellite uses innovative designs for attitude
determination and control (ADCS) and communications to deliver the
best performance for the lowest cost. A single, off-the-shelf ground
station transmits commands to the satellite and downloads images while
it is overhead. The real breakthrough with NanoObservatory is not
in capabilities but in cost. Weighing only 25kg, NanoObservatory can
ride as a secondary payload on other missions, lowering the cost of
launch. NanoObservatory takes advantage of the low radiation environment
in LEO by using commercial offthe- shelf parts in novel ways, which
lowers nonrecurring engineering costs. In addition, the satellite’s
simple command and control system requires only a basic ground station
and PC for operation, making its on-going costs a fraction of competing
earth imaging systems. Many science applications do not require highresolution
imaging, but could benefit from a low-cost, dedicated Earth imaging
platform. NanoObservatory fills this gap and delivers the benefits
of space-based remote sensing to a new segment of underserved customers.
NanoObservatory is a breakthrough technology that makes earth imaging
for everyone.


\subparagraph{Microsatellites at Very Low Altitude (2006- Session 2)\cite{Ref:Atir}}

An approach of using a very low flying orbit for a microsatellite
to achieve a low cost imaging mission, and its demonstration in the
Israeli - French VEN$\mu$S program scheduled to be launched in 2009, are
presented. At low satellite altitude, a smaller and less expensive
payload can be used to obtain the performance of another one that
operates at a higher altitude. This is true for the optical payload
(of the present concern), and is even much more conspicuous when an
RF payload is involved. More savings in mission cost are achieved
due to the impact of the payload size on the satellite and on the
launch cost. The main problem of flying at such altitudes is the orbital
rapid loss of energy due to the drag. This is compensated for by a
fuel efficient Hall Effect electrical thruster, and by proper low-drag
configuration design. The electrical thruster's high specific impulse
(of about 1350 sec. in our case) can provide the microsatellite with
several years of mission duration at the low altitude, using acceptable
amount of Xenon propellant. This concept is scheduled to be demonstrated
in the Israeli - French VEN$\mu$S (Vegetation and Environment New $\mu$Satellite)
program. The VEN$\mu$S mission is composed of two portionsa scientific
one and a technological one. The scientific mission consists of multispectral
Earth monitoring for vegetation and water resources quality, from
a Sun- Synchronous 2-day Earth repeating circular orbit, with a mean
altitude of 720 km. The technological mission has several goals, one
of which is to simulate a satellite flying in a high drag environment
and performing an enhanced mission. In the VEN$\mu$S program the enhanced
mission is the same scientific mission mentioned before, only that
it is carried out at a mean altitude of 410 km, thus providing a spatial
resolution almost twice as good, while maintaining the 2-day repeating
ground track. The required orbit corrections do not interfere with
the Earth monitoring task. The VEN$\mu$S platform, not designed to fly
at high air density environment, has over 2.7 m2 average cross section
area to the wind. Its mission at 410 km represents a mission at much
lower altitude (less than 300 km) of a different microsatellite, configured
especially for high drag environment. The paper describes the low
altitude microsatellite concept and analysis, the VEN$\mu$S mission with
emphasis on its technological portions, and the Israeli Hall Effect
thruster which is especially designed to be used by small satellites.
Main design issues, such as electrical power supply and environment
disturbances, are also addressed.


\subparagraph{Micro Satellite Bus for Stand-alone Earth Imaging / Space Science
Payloads (2006- Session 4)\cite{Ref:Murthy}}

ISRO has launched series of satellites for Earth Imaging for natural
resource applications. These applications are being served by various
large satellites. Some of the payloads with required swaths \& resolutions,
which can serve the applications separately in the areas of agriculture,
forestry, geology, water resources, land-use, infrastructure build-up,
pollution monitoring etc., can be launched on small or micro satellites
on stand-alone basis. With the experience of design, development,
fabrication and on-orbit operation of earth imaging satellites from
IRS-1A to Cartosat-1, the small/micro satellites are planned with
advanced technology. The small satellite project is envisaged to provide
platform for stand-alone payloads for earth imaging and science missions.
The current paper focuses on the first satellite. The satellite is
a three axis stabilized bus with two deployed solar panels. The bus
carries a miniature dual head star sensor, four micro reaction wheels
and micro gyros. The bus can operate in sun-pointing and earth pointing
modes and provides good attitude pointing and stability. The details
of mission, configuration, and the bus capability details are presented.


\subparagraph{VEN$\mu$S Program: Broad and New Horizons for Super-Spectral Imaging
and Electric Propulsion Missions for a Small Satellite (2008- Session
3)\cite{Ref:Herscovitz}}

Vegetation and Environment New Micro Satellite (Ven$\mu$s) is a joint
venture of the Israeli and French space agencies for development,
production, launching, and operating a new space system. Ven$\mu$s is
a Low Earth Orbit (LEO) small satellite for scientific and technological
purposes. The scientific mission includes vegetation monitoring and
water quality assessment over coastal zones and inland water bodies.
It will be specifically suitable for precision agriculture tasks such
as site-specific management and/or decision support systems. For this
purpose the satellite has apparatus for high spatial resolution (5.3
m) and for high spectral resolution (12 spectral bands in the visible
and near infrared wavelengths), as well as orbit for high temporal
resolution (2 days revisit time). The satellite's orbit is a near
polar sun-synchronous orbit at 720 km height. The satellite will acquire
images of sites of interest all around the world. The satellite will
be able to be tilted up to 30 degree along and across track; however,
each site will be observed under a constant view angle. The technological
mission consists of space verification and validation by mission enhancement
capability demonstration of a newly developed Israeli Hall Effect
Thruster (IHET) system, used as a payload. IHET is developed and manufactured
by Rafael and this will be its maiden flight. The heart of the IHET
is the HET-300 thruster, which produces about 15 mN thrust, operating
at 300W anodic power. This thruster and the based-on Electrical Propulsion
System (EPS), is specifically developed for usage onboard micro or
small satellites, which can supply as little as 300 to 600 watts for
operation. The technological mission will be targeted to qualify the
IHET in space as well as validate it by demonstrating orbit transfer
and strict orbit keeping in a high drag environment. The Ven$\mu$s satellite
is currently in manufacturing phase, its launch weight is 260 kg,
and it is planned to be launched in 2010. This paper will present
Ven$\mu$s system with emphasis on the two main missions (scientific and
technological) of Ven$\mu$s and the respective payloads along with main
design considerations of the electrical propulsion system.


\subparagraph{MISC – A Novel Approach to Low-Cost Imaging Satellites (2008- Session
10)\cite{Ref:Kalman}}

By severely limiting satellite size and weight, the popular CubeSat
nanosatellite standard realizes noticeable cost savings over traditional
satellites in the areas of design, manufacture, launch and operations.
To date, there has been limited commercial utilization of CubeSat
systems due to the widespread perception in industry that a 10 cm
x 10 cm x 30 cm form factor is too constrained for payloads in support
of useful missions. In this paper, we argue against this perception
by presenting MISC, a 3U CubeSat capable of providing 7.5 m GSD multispectral
imagery from a circular orbit of 540 km. Over an anticipated operational
lifetime of 18 months, each MISC will be able to image over 75 million
km2, equivalent to approximately half the Earth’s landmass. MISC’s
novel design combines a robust miniature imager module payload with
an existing CubeSat Kit-based bus and a distributed ground station
architecture. With anticipated order-of-magnitude cost savings when
compared to current commercial offerings, MISC's lifetime system cost
should represent an extremely attractive proposition to consumers
of satellite imagery that wish to own and operate their own assets.
MISC satellites will be available for commercial purchase in mid-2009.


\subparagraph{TINYSCOPE: The Feasibility of a 3-Axis Stabilized Earth Imaging CubeSat
from LEO (2008-Session 10)\cite{Ref:Blocker}}

The idea of a nano-sat for tactical imaging applications from LEO
is explored. On the battlefield, not every tactical situation requires
something as high-tech as an FA-18 dropping a GPS-guided weapon within
a couple of meters of the target to get the desired results - sometimes
a grenade or a mortar will do the trick. In the same way, a nano-sat
imaging from LEO may be a better solution than a national imaging
asset for some applications. These spacecraft may be used as short-term
low-cost independent elements, for instance; or perhaps in support
of traditional large imaging space systems as free-flying “targeting
telescopes”. They may also be deployed as elements of a LEO constellation
or cluster (think swarm), which would allow for quick re-targeting
opportunities over a large portion of the Earth. Tactical Imaging
Nano-sat Yielding Small-Cost Operations and Persistent Earth-coverage
(TINYSCOPE) is a preliminary investigation using analytical modeling
and laboratory experimentation to determine the potential performance
and the feasibility of using a 5-U CubeSat as an imager. Emphasis
is placed on three-axis attitude stabilization and slewing (for target
acquisition and tracking) and performance of various optics hardware
configurations. Numerical simulations will be conducted to support
the study, in particular on spacecraft dynamics and control.


\subparagraph{WNISAT - Nanosatellite for North Arctic Routes and Atmosphere Monitoring
(2010 - Session 2)\cite{Ref:Kim}}

WNISAT is a 10kg weight nanosatellite of the sun-synchronous low earth
orbit, currently being developed by Weathernews Inc. and AXELSPACE,
and waiting for the launch with PSLV of India until the year of 2011.
Weathernews Inc. is the largest private weather service company headquartered
in Japan, which have many sea liners as their customers. Currently,
the north arctic ice region reaches the lowest level every year, and
sea liners have great interest for this because the north arctic routes
means very short distance compare to other routes. Many approaches
have been suggested for the monitoring of north arctic routes, and
constellation of small satellites is one of the best ways considering
efficiency. From this reason, Weathernews Inc, decided to develop
small satellites in close collaboration with AXELSPACE. AXELSPACE
is a university venture company established in 2008, all engineers
have considerable experience in the field of small satellites of nano-class
through many projects of their universities. There are two major missions
for WNISAT, the first mission is earth observation of commercial use
as like explained already. It is challenging to provide with the ice
coverage information over high latitude oceans including NIR spectral
ranges. And the second mission is atmosphere monitoring for environmental
issue, the density of carbon dioxide in atmosphere using a laser application.
The bus system of spacecraft and the first mission development are
being led by AXELSPACE. The laser application for the second mission
is under development by Weathernews Inc. The object of this WNISAT
is to show the feasibility of nanosatellite for two major missions,
especially the commercial use of small satellites of nano-class. After
this WNISAT, several satellites are scheduled for the practical and
commercial use within three or five years. At first, this paper will
review progress in the development of WNISAT. And, the entire structure
of spacecraft and the sub-systems are presented for the review and
the detail explanation. After that, it will review the relevance of
WNISAT's technology to advanced sensing concepts, reliable and efficient
remote sensing and issues of atmospheric carbon dioxide content monitoring.
Finally, future schedule after WNISAT is also briefly presented.


\subparagraph{NovaSAR: Bringing Radar Capability to the Disaster Monitoring Constellation
(2012-Session 1)\cite{Ref:Davies}}

Small satellites are playing an increasing role in addressing applications
in Earth Observation for scientific, civil and military applications.
With all optical systems, this leads to some obvious limitations to
the time of day targets can be imaged, which geographic latitudes
can be covered, and to a dependency on cloud cover. For some applications
this limits the utility of space systems unless low-light and through-cloud
imaging information can be obtained in a timely fashion by other cost-effective
means. Typically, space based radar systems are significantly more
complex, more expensive, and data is more difficult to utilise than
equivalent optical systems. Existing radar satellite systems therefore
predominantly address scientific and military needs, leaving room
for smallsat systems that address commercial needs, maritime security,
and disaster monitoring. Advances in new technologies now have permitted
a step performance improvement in radar systems, which will now be
implemented in the NovaSAR mission which is under construction at
SSTL. Gallium Nitride RF transistors enable high efficiency power
amplifiers to be employed, reducing the power demand from solar panels,
thus enabling a smaller radar satellite to be constructed. System
innovations are also included to facilitate satellite operation in
constellations, and in orbits other than the traditional dawn-dusk
orbits. The spacecraft will also include an operational mode to operate
in a maritime detection mode instead of imaging mode. In November
2011 the UK government announced that they are investing in the first
satellite in a NovaSAR radar constellation, allowing the construction
of the first 400kg satellite to commence to be ready for launch in
2014. This paper will provide details on the satellite and payload
design trades, results from airborne trials of the payload, and provides
an overview of the planned mission applications.


\subparagraph{Commissioning of the NigeriaSat-2 High Resolution Imaging Mission
(2012- Session 11)\cite{Ref:Curiel}}

The manufacture of the NigeriaSat-2 spacecraft was completed in 2010,
and was successfully launched in August 2011. This is a state-of-the-art
small satellite Earth observation mission including several innovations
not previously seen on small spacecraft, which will provide high duty
cycle imaging of the Earth in high resolution. It will be used by
the Nigerian government for mapping and to monitor a number of environmental
issues within the country. The key requirements of this mission are
to provide high volume mapping data, coupled with highly accurate
image targeting and geolocation, and sufficient agility to enable
a wide range of complex operational modes. This paper focuses on the
challenges associated with designing a spacecraft system that can
meet these requirements on a satellite with a mass of less than 270kg.
The paper will describe how the stereo, mosaic and other imaging modes
can be employed using the agility of the spacecraft. Inertia calibration
and on-board navigation techniques used to give the required targeting
accuracy are discussed, and the interaction between the attitude control
system and the mechanical design is detailed. The payload isolation
system used to ensure image quality and geolocation performance is
also described. An overview of the final test and launch campaign,
and first in-orbit results from the satellite commissioning are provided.


\subsection{Pinpointing the Source of Gamma Ray Bursts}


\subsubsection{Mission Concept}

The formation could be used to source GRBs through precise triangulation.
If we could get the measurements of inter-satellite distance and GRB
incident time accurately enough, we could potentially increase the
accuracy of GRB detection and positioning. 


\subsubsection{Abstract}


\subparagraph{The Space System for the High Energy Transient Experiment (1992-Session
2)\cite{Ref:Dill}}

The High Energy Transient Experiment (HETE) is an astrophysics project
funded by NASA and led by the Center for Space Research (CSR) at the
Massachusetts Institute of Technology (MIT). It has for principal
goal the detection and precise localization of the still mysterious
sources of gamma ray bursts. The project is original in many respects.
HETE will provide simultaneous observations of bursts in the gamma,
X-ray and UV ranges from the same small (250 Ibms) space platform.
A network of ground stations around the world will diffuse in real
time key information derived from HETE observations to many ground
observatories, allowing quick follow-on observations with ground instruments.
The whole project is entirely managed by MIT, under top level NASA
supervision, and satellite and ground stations will be remotely operated
from CSA. The HETE system development is conducted with a small budget
and under a short schedule.


\subsection{Observing Gamma Rays Emitted by Thunderstorms}


\subsubsection{Mission Concept}

Look for gamma rays emitted by thunderstorms. NASA's Fermi telescope
has observed the phenomenon, but it could possibly do with more study
and happen in conjunction with the GRB monitoring. {[}http://www.nasa.gov/mission\_pages/GLAST/news/fermi-thunderstorms.html{]}


\subsubsection{Abstract}


\subparagraph{A Small/ Micro/ Pico Sat Program for Investigating Thunderstorm-
Related Atmospheric Phenomena (1998- Session 7)\cite{Ref:Kitts}}

A low cost, multi-satellite program for conducting novel atmospheric
science is described. The program exploits simple visual and radio
frequency sensing in order to investigate atmospheric phenomena induced
by thunderstorms. The science instrumentation is supported by a variety
of standalone and collaborative small, micro-, and picosatellite vehicles
in order to meet the program's science requirements while also exploring
a spectrum of small satellite approaches for conducting meaningful
science. This multi-mission program is being developed jointly by
student design teams at Stanford University and Santa Clara University.


\subsection{Tracking Asteroids and Satellite Debris }


\subsubsection{Mission Concept}

Search for Earth-Approaching Asteroids and potentially hazardous debris
satellites. Could act as a guard dog for the ISS. 


\subsubsection{Abstract}


\subparagraph{NESS: Using a Microsat to Search for and Track Satellites and Asteroids
(2000- Session 2)\cite{Ref:Carroll}}

The Near Earth Space Surveillance (NESS) mission is being developed
by Dynacon and a team of asteroid scientists, supported by the Canadian
Space Agency (CSA) and the Canadian Department of National Defence
(DND). NESS uses a single satellite to perform a dual mission: searching
for an tracking Earth-approaching asteroids, and tracking satellites
in Earth orbit. There are aspects of both of these activities that
are best accomplished using an orbiting observatory. The concept presented
here is to implement NESS using a small imaging telescope mounted
on a lowcost microsatellite-class platform, based on the design developed
for the MOST stellar photometry microsatellite mission.


\subparagraph{Space-Based Radar to Observe Space Debris (1998- Session 12)\cite{Ref:Tolkachev}}

Space debris of 1\textdiv{}3mm size is known to be hazardous for astronauts
and space vehicles. At the same time the possibility to notice such
objects by ground based optical and radar devices in the nearest future
is rather problematic. Here we propose an idea of space radar for
observation of cicumterrestrial space debris. The radar works in short
wave part of millimetre band, which is mostly suitable for this purpose.
The radar provides detecting and tracking of 1mm size objects within
40000m2 area. The radar antenna is Phased Array Antenna with 2m diameter.
The radar's weight is about 100kg, the consumed power - 2.5kW.


\subparagraph{How Can Small Satellites be used to Support Orbital Debris Removal
Goals Instead of Increasing the Problem? (2010- Session 2)\cite{Ref:Guerrero}}

Orbital debris is a serious concern for the NASA, DARPA, Air Force
organizations and the commercial space industry. Since 2005, the space
debris environment has been unstable and began a collision cascade
effect per NASA. A recent International Orbital Debris Conference
focused on the need to find solutions for Orbital Debris Removal and
manage any space debris increase potential. The purpose of this paper
is to explore what orbital debris issues can be address by Small Satellites.
The paper will discuss a technology supported by Small Satellites
to resolve the Orbital Debris problem. It will concentrate on mitigation
of debris sizes from 1cm to 10cm, which are unable to be tracked by
current ground systems capabilities but can cause serious damage or
destroy spacecraft. Requirements will be for the LEO orbit, where
there are known significant numbers of debris of this size. A Method
will be proposed, by which a small spacecraft can be used to sweep
volumes of specific orbits to remove or collect debris.


\subparagraph{Sapphire: A Small Satellite System for the Surveillance of Space
(2010- Session 2)\cite{Ref:Leitch}}

The tracking of man-made objects in Earth orbit is a crucial function
of the Canadian Space Surveillance System (CSSS). This system will
contribute information to the United States Space Surveillance Network
(SSN) which maintains a global catalog of orbit elements for Resident
Space Objects (RSOs). RSOs include active and inactive satellites,
spent rocket bodies, and other pieces of orbital debris created by
decades of human activity in space. Sapphire is a small satellite
system that will form the centerpiece of the CSSS, providing an operationally
flexible space-based platform for the precise tracking and identification
of RSOs covering orbit altitudes in the range from 6000 km to 40000
km. The Sapphire system, including a satellite, ground segment, launch,
and operations, is currently being developed by MDA for the Canadian
Department of National Defence (DND), with satellite launch scheduled
for 2011. This paper describes the Sapphire design. Sapphire must
meet demanding performance requirements for RSO detection and pointing
determination accuracy as well as system responsiveness and imaging
task throughput. Sapphire will provide continuous service over a mission
life of at least five years. The paper discusses the approaches used
to build a robust capability into a small satellite package, including
the extensive use of flight-proven heritage in the satellite subsystems.
In addition, the paper discusses the role of the satellite with respect
to the ground system elements and summarizes some of the major system-level
tradeoffs from the design process.


\subsection{Monitoring Atmospheric Plasma Depletion to Predict Outages in GPS
and Communications}


\subsubsection{Mission Concept}

Depletion in ionospheric plasma can disrupt signal transferrance,
and not much is known about the depletion zones. The formation would
study how the depletions change and propagate so that scientists can
create a model and further their understanding of the phenomenon.


\subsubsection{Abstract}


\subparagraph{Target of Opportuinty Multipoint in Situ Measurements with Falconsat-2
(2002- Session 9)\cite{Ref:Krause}}

This paper describes the FalconSAT-2 mission objectives to take advantage
of targets of opportunity to make multipoint in situ measurements
of ionospheric plasma depletions simultaneously with other spacecraft.
Because these plasma depletions are known to interfere with radio
transmissions over a broad range of frequencies, including 100-1000
MHz, the international space weather community is investigating the
instigation, temporal evolution, and spatial propagation of these
structures in the hopes that a prediction tool may be developed to
warn operators of outages in communications or navigation. FalconSAT-2
will be launched into a low altitude (360 km), medium inclination
(52 degrees) orbit with sensors designed to measure in situ suprathermal
plasma spectra at a rate of 10 samples per second. The primary mission
objectives are to 1) investigate F region ionospheric plasma depletion
morphology relative to geomagnetic activity, and 2) demonstrate the
utility of the Miniature Electrostatic Analyzer (MESA) in measuring
energy-resolved spectra of ionospheric electrons over a dynamic range
such that plasma density depletions down to 0.1\% of the background
may be resolved at a rate of 10 Hz. Simultaneous in situ multipoint
observations of ionospheric plasma depletions are designated as a
secondary objective since FalconSAT-2 consists of a single spacecraft,
and opportunities to make these simultaneous measurements with other
spacecraft in compatible orbits are not in our control. Both deep
and shallow bubbles, frequently observed in the pre- and post-midnight
sectors, respectively {[}Singh at al., 1997{]}, are known to exhibit
magnetic field-aligned behavior {[}Fagundes et al., 1997{]}; thus,
there is the expectation (to first order) that multiple spacecraft
entering a magnetic flux tube simultaneously have the opportunity
to observe a depletion structure at different points within the structure.
This observation would provide insight into the plasma depletion extent
along the field line. Other conjunction types, such as non-simultaneous
intersection of a flux tube or crossing of orbital paths simultaneously
in different magnetic flux tubes, provide insight into other aspects
of depletion structure, such as constraining the plasma depletion
extent and propagation speed along the magnetic field line, or plasma
depletion vertical extent. With this paper, a statistical analysis
of the probability that FalconSAT-2 will intersect a magnetic flux
tube during eclipse simultaneously with other spacecraft capable of
measuring thermal electrons is presented.


\subparagraph{Low-Resource CubeSat-scale Sensorcraft for Auroral and Ionospheric
Plasma Studies (2010- Session 1)\cite{Ref:Bracikowski}}

Explicitly separating variations in space from variations in time
over a large volume is a current unmet challenge for in situ studies
of the ionosphere and aurora. We propose that arrays of many (\_ 10)
low-resource sensorcraft can address this scientific and technical
challenge. We are developing a suborbital CubeSat, RocketCube, to
enable low-cost multipoint measurements for orbital and sub-orbital
scientific missions. The graduate student-designed RocketCube showcases
a new scientific instrument, the Petite Ion Probe (PIP), and an FPGA-based
instrumentation and payload bus system designed specifically with
the ionosphere in mind. The PIP, a retarding potential analyzer, measures
thermal plasma parameters to characterize the ionosphere. In addition
to control and data handling, RocketCube’s bus system will allow synchronization
of PIP activity between payloads in an array to the order of \_1 $\mu$s
from timing provided by a qualified GPS. As of this writing (June
2010), RocketCube may be repackaged and manifested as a deployable
subpayload on the Cornell University MICA (Magnetosphere-Ionosphere
Coupling in the Alfvn Resonator) mission scheduled for a winter 2012
sounding rocket launch. Additionally, RocketCube is enabling us to
be currently proposing our next scientific sounding rocket mission,
called Probe Array Lattice to Investigate Spatial Auroral DEnsity
Structuring (Palisades), to NASA’s G/LCAS (Geospace Low Cost Access
to Space) program. Palisades will feature an array of 12 subpayloads
containing our bus system and two PIPs per payload to study the auroral
driving of the ionosphere. This paper provides an overview of RocketCube’s
purpose, design, and current status including details of the PIP instrument.


\subparagraph{Sensitivity of Ionospheric Specifications to In Situ Plasma Density
Observations Obtained From Electrostatic Analyzers Onboard of a Constellation
of Small Satellites (2012- Session 4)\cite{Ref:Blalthazor}}

Our ability to specify and forecast ionospheric dynamics and weather
at low- and mid-latitudes is strongly limited by our current understanding
of the coupling processes in the ionosphere-thermosphere system and
the coupling between the high and low latitude regions. Furthermore
only a limited number of observations are available for a specification
of ionospheric dynamics and weather at these latitudes. As shown by
meteorologists and oceanographers, the best specification and weather
models are physics-based data assimilation models that combine the
observational data with our understanding of the physics of the environment.
Through simulation experiments these models can also be used to study
the sensitivity of the specification accuracy on different arrangements
of observation platforms and observation geometries and can provide
important information for the planning of future missions. For example,
these studies can provide information about the number of spacecraft
needed to improve the specification or evaluate the impact of different
observation geometries on the accuracy of the specification. Here
we have used the Global Assimilation of Ionospheric Measurements Full-Physics
model (GAIM-FP) to study the sensitivity of ionospheric specifications
on in situ plasma density observations obtained from electrostatic
analyzers (ESA) onboard of a constellation of small satellites. The
model is based on an Ensemble Kalman filter technique and a physics-based
model of the ionosphere/plasmasphere (IPM), which covers the altitude
range from 90 to 20,000 km. The data assimilation model can, in addition
to the ESA observations, assimilate bottom-side Ne profiles from ionosondes,
slant TEC from ground-based GPS stations, in situ Ne from DMSP satellites,
occultation data from several satellites, and line-of-sight UV emissions
measured by satellites. Simulation studies have been performed using
various ESA constellation arrangements and their impact on the ionospheric
specification has been evaluated. The results from this study will
be presented with an emphasis on the number of satellites and their
orbital geometries needed to improve ionospheric specifications and
forecasts.


\subsection{Space-Based Ocean Monitoring}


\subsubsection{Mission Concept}

Monitor the health of the oceans, rivers, lakes, etc, through multi-spectral
imaging. Better understand the effects of tides on ocean color. The
mission described is extremely aggressive, ours would be much much
smaller. 


\subsubsection{Abstract}


\subparagraph{‘Charybdis’ – The Next Generation in Ocean Colour and Biogeochemical
Remote Sensing (2012- Session 4)\cite{Ref:Lowe}}

Within the field of Space-based Maritime observation, there exists
an opportunity in the form of high spatial, high temporal resolution
multi-spectral imaging to map coastal and inland waterway colour and
biogeochemistry. Information provided would help environmental agencies
and the scientific community to better understand patterns and evolution
of ecological systems, sediment suspension in river estuaries and
the effects of anthropogenic processes on our water systems. In addition,
monitoring of these colour patterns with respect to the well understood
tidal sequence would provide significant benefits to our understanding
of the way in which tidal forcing affects ocean colour. This paper
describes the astrodynamic properties of a tidal-synchronous satellite
trajectory and the system-level design of a multi-platform CubeSat
constellation capable of high resolution, multispectral imaging. The
constellation, named ‘Charybdis’, is envisaged to be dedicated to
providing unprecedented levels of data (high temporal and spatial
resolution) of coastal regions and inland waterway colour and biogeochemistry.
Analyses of two alternative missions are presented; one providing
bi-hourly, global coverage from 115 nanosatellites and a second providing
bi-hourly regional coverage over the UK mainland from 30 nanosatellites.


\subsection{Interferometry and Synthetic Aperture Radar Formation Flying}


\subsubsection{Mission Concept}

A formation of >2 CubeSats will work together to create a digital
terrain model


\subsubsection{Abstract}


\subparagraph{Possible Orbit Scenarios for an InSAR Formation Flying Microsatellite
Mission (2008- Session 6)\cite{Ref:Peterson}}

Multistatic interferometric synthetic aperture radar (InSAR) is a
promising future payload for a small satellite constellation, providing
a low-cost means of augmenting proven “large” SAR mission technology.
The Space Flight Laboratory at the University of Toronto Institute
for Aerospace Studies is currently designing CanX-4 and CanX-5, a
pair of formation-flying nanosatellites slated for launch in 2009.
Once formation flight has been demonstrated, a future multistatic
InSAR formation-flying constellation can exploit sub-centimeter inter-satellite
baseline knowledge for interferometric measurements, which can be
used for a myriad of applications including surface deformation, digital
terrain modeling, and moving target detection. This study evaluates
two commonly proposed InSAR constellation configurations, namely the
Cartwheel and the Cross-Track Pendulum, and considers two ‘large’
(\textasciitilde{}kilowatt) SAR transmitters (C- and X-band) and one
microsatellite transmitter (X-band, 150W). Each case is evaluated
and assessed with respect to the available interferometric baselines
and ground coverage. The microsatellite X-band transmitter is found
to be technically feasible, although the lower available transmitter
power limits the operating range. The selected transmit band determines
the maximum allowable cross-track baseline between receiver satellites
in the constellation. Additionally, the Cartwheel and Cross-Track
Pendulum configurations offer different available baselines and ground
coverage patterns, namely, the Cartwheel eliminates the near-zero
cross-track baseline component that contributes to DEM height errors
but adds a coupled along-track baseline, while the Cross-Track Pendulum
offers the advantage of independent cross-track and along-track baseline
components. Ultimately, the primary application for the InSAR data
will dictate the transmit band used, the desired baselines, and the
receiver constellation configuration.


\subparagraph{WiSAR: A New Solution for High-Performance, Smallsat-Based Synthetic
Aperture Radar Missions (2008- Session 6)\cite{Ref:Fox}}

To date, high performance Synthetic Aperture Radar (SAR) satellites
have typically featured massive power generation, storage and distribution
subsystems, together with complex, heavy and rigid deployable SAR
antenna structures. The result is an expensive, heavy satellite. MDA’s
Wireless Synthetic Aperture Radar (WiSAR) is a new SAR payload design
that offers leading-edge performance in both X- and C-band at a significantly
lower cost than the conventional state of the art. The key technology
breakthrough is a modular, low mass phased array antenna technology
that enables high performance multi-mode SAR imaging from a smallsat
platform. The WiSAR solution uses proven off-the-shelf technologies
from the automotive and terrestrial wireless communication industries
to enable an innovative space-fed active lens architecture that replaces
the heavier and bulkier constrained feed design of traditional high
performance SAR payloads. Key elements of the WiSAR payload include:
self-contained active antenna nodes; low cost RF radiators; thin,
lightweight, easily deployed antenna panels; and RF ranging for dynamic
antenna distortion compensation. This paper describes the WiSAR™ payload
technology in various configurations; a High Resolution X-Band Smallsat
SAR, a Dual Aperture X-Band GMTI SAR, a High Performance C-Band SAR,
and a C-Band Smallsat SAThe results of a project to build, test and
operate a fully functioning C-band prototype WiSAR phased array antenna
are presented, along with current developments in X-Band.


\subsection{Testing Satellite Tether Deployment and Operations}


\subsubsection{Mission Concept}

Hardware test for satellite tether, can be used to create artificial
gravity to aide in long term human missions. 


\subsubsection{Abstract}


\subparagraph{The Kyushu/US Experimental Satellite Tether (QUEST) Mission, A Small
Satellite to Test and Validate Spacecraft Tether Deployment and Operations
(2000- Session 7)\cite{Ref:Carlson}}

In recent years, an increased effort to design, build, and operate
small satellites has taken place in universities and laboratories
all over the world. These microsatellites provide numerous flight
opportunities for science experiments at a fraction of the cost of
larger traditional missions. In addition, there has been an increasing
trend towards international cooperation on space projects. From the
International Space Station to joint commercial ventures, the future
of space progress will be shared by countries around the world. Tomorrow’s
engineers must prepare for this challenge. This paper provides an
overview of the Kyushu/US Experimental Satellite Tether (QUEST) mission,
a joint project between Kyushu University (KU), Arizona State University
(ASU), and Santa Clara University (SCU). This mission will develop
and test new technologies related to space tether deployment and operation.
In particular, it will attempt to show very small space platforms
can be used for significant tether deployments. If successful, it
will provide valuable data for tether designers as well as cost and
weight savings on future missions. In addition, progress on system
design, ground station development, orbital simulations and related
testing are reviewed.


\subparagraph{A Small-Satellite Demonstrator for Generating Artificial Gravity
in Space via a Tethered System (2002- Session 3)\cite{Ref:Mazzoleni}}

It is well-known that prolonged exposure in humans to a microgravity
environment leads to significant loss of bone and muscle mass; this
presents a formidable obstacle to human exploration of space, particularly
for missions requiring travel times of several months or more, such
as a 6 to 9mon th trip to Mars. Artificial gravity may be produced
by spinning a spacecraft about its center of mass, but since the g–
force generated by rotation is equal to “omega-squared times r” (where
omega is its angular velocity and r is the distance from the center
of rotation), we have that unless the distance from the center of
rotation is several kilometers, the rotation rate required to generate
“1 - g” would induce vertigo in the astronauts as they moved about
the capsule (e.g. if the distance from the center of rotation is 10
meters, the required rotation rate for 1 - g would be 9.5 rpm). By
tethering the crew capsule to an object of nearly equal mass (such
as the spent final rocket stage) at a distance of 1 to 2 kilometers,
the necessary rotation rate would be sufficiently small as to not
cause discomfort for the astronauts. For example, if the distance
from the center of rotation is 2 kilometers, the required rotation
rate for 1-g would be 0.67 rpm; at 1 kilometer the rate is still only
0.95 rpm. 1 rpm is considered an acceptable spin rate for the human
body to withstand for extended periods of time. This paper gives an
overview of the Tethered Artificial Gravity (TAG) satellite program,
a 2-part program to study the operation and dynamics of an artificial-gravity-generating
tethered satellite system. Phase I of the program will culminate in
a flight of a model spacecraft in a non-ejected Get-Away-Special (GAS)
Canister on the Space Shuttle. It is to be operated under the aegis
of the Texas Space Grant Consortium. The purpose of the Phase I flight
is to test key components of the system to be flown in Phase II of
the program. Phase I will also involve detailed modeling and analysis
of the dynamics of the spacecraft to be flown in Phase II of the program;
the Phase II spacecraft will be a small, 65 kg, tethered satellite
system which will be boosted into low-earth orbit, deployed and then
spun-up to produce artificial gravity. In addition to a description
of the TAG program, results of parametric studies related to TAG will
be presented in this paper.


\subsection{Studying Sub-dwarf Stars Using a Small Telescope}


\subsubsection{Mission Concept}

Use the satellites to study distant sub-dwarf stars. Formation seems
unnecessary. 


\subsubsection{Abstract}


\subparagraph{MOST- Microsat Mission: Canada's First Space Telescope (1998- Session
6)\cite{Ref:Carroll2}}

The MOST (Microvariability and Oscillations of STars) astronomy mission
has been chosen by the Canadian Space Agency's Small Payloads Program
to be Canada's first space science microsatellite, and is currently
planned for launch in late 2001. The MOST science team will use the
MOST satellite to conduct long-duration stellar photometry observations
in space. A major science goal is to set a lower limit on the age
of several nearby \textquotedbl{}metal-poor sub-dwarf\textquotedbl{}
stars, which may in turn allow a lower limit to be set on the age
of the Universe. To make these measurements, MOST will incorporate
a small (15 cm aperture), high-photometric-precision optical telescope
to be developed by UBC. The MOST bus and ground stations are being
developed by Dynacon and the University of Toronto, in collaboration
with AMSAT Canada. Several of the bus subsystems are based on similar
designs that have been flown on past AMSAT microsatellites. However,
the MOST attitude control system is unusual for a microsatellite,
requiring highly-accurate (\textless 30 arc-seconds) three-axis inertially-fixed
stabilization, far better than can be achieved using the gravity-gradient
boom stabilization approach typical of many past microsatellites.
Dynacon will provide the MOST ACS, based on its Miniature Reaction
Wheel (MRW) and High Performance Attitude Control (HPAC) products.
MOST's HPAC capability will enable it to be one of the first operational
space science microsatellites.


\subparagraph{DISC Experiment Overview and On-Orbit Performance Results (2012-
Session 11)\cite{Ref:Nicholas}}

The Digital Imaging Star Camera (DISC) experiment has successfully
imaged star fields from the International Space Station (ISS). DISC
is a Naval Research Laboratory (NRL) led payload developed jointly
by NRL and the Utah State University Space Dynamics Laboratory (SDL)
to advance miniaturized technology for accurate precision pointing
knowledge in space which is a critical mission requirement for many
scientific and operational payloads. The low size, weight and power
(\textless10x10x10 >cm, \textless1kg, \textless1 W) sensing platform that will provide an
enhanced pointing capability for nano- and pico- satellite busses.
It is flying on the ISS as part of the Air Force Space Test Program
STP-H3 flight to provide a proof of concept for DISC experiment. This
technology represents a key transition from large, high cost, long-timescale
programs to small, low-cost, rapid response science enabling sensing
platforms. This paper will focus on the instrument design and on-orbit
mission performance.


\subsection{Completing the Map of the Earth's Electric Field}


\subsubsection{Mission Concept}

This project would use a system of small satellites to observe the
Earth's electric field with radar measurements. It would appear that
other systems already make these measurements, but there are still
areas of poor coverage that could be addressed {[}http://www.sciencedirect.com/science/article/pii/S1364682612002866{]}. 


\subsubsection{Abstract}


\subparagraph{Microspacecraft and Earth Observation: The Electrical Field Measurement
Project (1990- Session 8)\cite{Ref:Redd}}

Past attempts to map the earth's electrical field have been severely
limited by the lack of simultaneous global measurements. Previous
measurements have been made by sounding rocket and satellite borne
sensors, but these measurements have covered only singular points
in the field. These satellite observations are augmented by ground
radar (incoherent scatter) plasma drift measurements; however, only
six ground-based installations are producing such local electrical
field maps. The expansion of this ground-based radar network to meet
a global objective is politically and financially impossible. Global
electrical field maps constructed by forcing mathematically formulated
models to fit this limited set of data points are not only inaccurate,
but the degree of inaccuracy is impossible to evaluate. This paper
discusses the design of a global electrical field sensing system to
be deployed in a constellation of microspacecraft. Each microspacecraft
incorporates a deployable sensor array (5 m booms) into a spinning
oblate platform. Global deployment of 48 spacecraft is achieved through
perturbation-driven dispersion of multiple spacecraft launched from
eight Pegasus launch vehicles. The mass of each spacecraft is less
than 25 kg, and the power requirements are less than 10 W; all the
required power can be generated by solar cells covering the exterior
of the spacecraft. The program costs are estimated to be less than
\$100 million.


\subsection{Raman Spectroscopy to investigate the atmosphere}


\subsubsection{Mission Concept}

Create a more in-depth model of the upper atmosphere using Raman Spectroscopy
from multiple sources in a formation to achieve a 3D (or at least
more comprehensive) map. The main obstacle here would be finding a
detector array that would fit in our size constraints. 


\subsubsection{Abstract: }


\subparagraph{Raman Spectroscopy to investigate the atmosphere (1988- Session 3)\cite{Ref:Cantrell}}

Raman Spectroscopy is an active remote sensing method that can map
planetary mineral and chemical abundances and their distributions.
Raman spectroscopy can also be used to study the chemical composition
of various planetary atmospheres. The remote raman technique utilizes
a low power laser to stimulate raman scattering at the substance and
a spectrometer receives the returned raman spectrum at the spacecraft.
The returned spectrum contains the shifts in frequency, shifted from
that of the incident laser light, that are characteristic of the substance.
The intensity of the raman spectrum lines is proportional to the amount
of the substance present. Thus, with raman spectral information, the
identification of a substance and an estimate of its volumetric concentrations
can be achieved. The baseline remote raman instrument system utilizes
a 10 W krypton laser and a RIRIS spectrally sensitive detector array,
cassegran optics, has a mass of 200 Kg, and consumes 1 KW of electrical
power. This paper explores the basic concepts of remote raman spectroscopy
and postulates an instrument package that is compatible with the mass
and size constraints of a small satellite. Various solar system exploratory
missions using raman spectroscopy are discussed including the study
of the surface of the moon, Earth's upper atmosphere, the atmosphere
of Mars, the atmosphere of Jupiter, and the rings of Saturn.


\subsection{Formation Flying to Sample Volume of Magnetosphere}


\subsubsection{Mission Concept}

Use the formation of cubesats to create a more detailed 3D model of
the magnetosphere. This may not be necessary because of the twin satellites
launched by NASA in Fall 2012. Also several earlier missions acheived
similar results, not using formations but I doubt the information
is still needed. {[}http://www.ucl.ac.uk/news/news-articles/1009/10090204{]} 


\subsubsection{Abstract}


\subparagraph{Picosat Free Flying Magnetometer Experiment (1996- Session 2)\cite{Ref:Clarke}}

Individual satellites have been measuring the Earth's magnetic field
since 1958. Measurements taken in this way have led to some interesting
discoveries about the earth's magnetosphere. However, they have also
raised many questions about the magnetosphere's finer texture and
dynamic nature. Researchers at JPL have proposed a mission where a
single larger satellite ejects several picosatellites in order to
simultaneously sample a volume of space. Each picosat is to carry
a small, two axis, fluxgate magnetometer, several photo detectors
for spin rate detection, a micro processor and a high frequency transmitter.
After launch from the main satellite, each picosat will transmit its
sensor readings back to the main satellite where the data will be
stored for retrieval. Issues addressed in this paper are related to
the design, manufacture, and planned flight test of the picosatellite
on OPAL, a Stanford University Student Spacecraft


\subsection{Formation Flying Educational Platform}


\subsubsection{Mission Concept}

This is essentially what we are doing. We would test out our formation
flying algorithms, then open the platform to outside universities
or scientists to test their algorithms. 


\subsubsection{Abstract}


\subparagraph{Design of Small Satellite for Use in Astronautics Education (1988-
Session 2)\cite{Ref:Daniel}}

The Naval Academy is pursuing a small satellite project to give midshipmen
hardware and ground control experience. The concept is for a lightweight,
gravity-gradient stabilized satellite to be deployed from a Get Away
Special cannister; the payload would be two small, low-resolution
imagers (one visible and one infrared). The 40 foot diameter ground
station antenna being installed at the Naval Academy can receive a
power of 10 milliwatts from a satellite that uses an omnidirectional
antenna and is in a typical Space Shuttle orbit. Approximately 5 to
7 watts of power is constantly available to the payload of up to 10.4
Ib (4.7 kg). Thermal control is largely passive, but some heating
will be required during the longest eclipses. Once in orbit, the spacecraft
would be used in astronautics courses for such assignments as orbit
determination, decoding telemetry, and processing images.

%\bibliographystyle{IEEEtran}
%\bibliography{Bibliography_Formation_Flying_Saptarshi,articles}

%\end{document}
